\section{Data and Methodology}
%THIS SHOULD DESCRIBE THE SATELLITE DATA AND THE TECHNIQUES USED TO PROCESS AND ANALYSE THESES DATA.
\subsection{Observations and Data Reduction}
Using data collected by spacecraft observing the Sun, energy released during solar flares can be tracked as it is deposited throughout the atmosphere. The X1 flare of the 29th of March 2014 occurred in active region NOAA 12017, which contained two sunspots in a bipolar configuration. The sunquake was at heliocentric coordinates, 518", 262", which fortunately is a region that many spacecraft were observing at the time. Included in those observing were RHESSI, IRIS and SDO/HMI, collecting HXR, UV and optical emission respectively. The peak of the impulsive phase of the flare occurs at 17:47 UT, at which point all mentioned instruments provide good coverage. The associated sunquake impact is calculated to have area $A_{sqk} \sim 2.6{\times}10^{16}$ cm$^{2}$ and power, $P_{sqk} \sim 1.3\pm0.05{\times}10^{26}$ erg s$^{-1}$ \citep{2014ApJ...796...85J}.

\subsubsection{The Ramaty High Energy Solar Spectroscopic Imager}\label{rhessi}
%insert RHESSI background info
The Ramaty High Energy Solar Spectroscopic Imager (RHESSI) observes solar emission ranging from 1 keV X-rays to 20 MeV $\gamma$-rays produced by energetic particles and nuclear interactions. RHESSI was designed with the aim of understanding impulsive energy release, particle acceleration and transportation in the magnetohydrodynamic environment of the solar atmosphere. Isolating the 10 - 100 keV energy data collected by RHESSI can provide information regarding the intensity and spatial origin of a HXR source. This allows the location of magnetic HXR footpoints to be tracked and the calculation of energy deposition by accelerated electrons. 
  
Applying a thick target model fit to RHESSI data is achieved by using the \texttt{ospex} software within SolarSoft (SSWIDL). The entire data set has to be split into short intervals to improve the temporal resolution of fitting, hence increasing the detail seen in the time evolution of the signal. The attenuator state of the instrument has to be taken into account due to differences in sensitivity to incoming photons. Therefore it is important to define intervals for fitting first by the attenuator state as otherwise, \texttt{ospex} will not mitigate for the differences in count sensitivity. Then each attenuator time period can be split further in to smaller time increments. The chi squared statistic $\chi^{2}$ for fitted RHESSI data can be seen in Figure \ref{chi} in Section \ref{rhessiresults}

\subsubsection{The Interface Region Imaging Spectrograph}
Observing UV ribbons requires a different spacecraft. The Interface Region Spectroscopic Imager (IRIS) captures near-ultraviolet (NUV) and far-ultraviolet (FUV) emission and is designed to observe the chromosphere and transition region at various altitudes. Emission is collected by a slit-jaw imager (SJI) and a spectrometer (SG) simultaneously. The spectrograph is sensitive in both FUV and NUV passbands, which expose 3 CCDs to produce spectra in three UV bands, two FUV and one NUV. Table \ref{iris-sg} shows how each passband relates to emission processes occurring from the upper-chromosphere down to the upper-photosphere.

\begin{table}[H]
\centering
\begin{tabular}{|c|c|c|c|}
Band & Wavelength \AA\ & Temperature $\log{T}$ & Region of Atmosphere\\
\hline
FUV 1 & $1331.7 - 1358.4$ & $3.7 - 7.0$ & Upper to lower-chromosphere\\
FUV 2 & $1389.0 - 1407.0$ & $3.7 - 5.2$ & Upper to lower-chromosphere\\
NUV & $2782.7 - 2851.1$ & $3.7 - 4.2$ & Chromosphere to upper-photosphere\\
\end{tabular}
\caption{The IRIS/SG is capable of observing three passbands, which relate to different plasma temperatures.}\label{iris-sg}
\end{table}


The slit-jaw images, are light collected from a reflective area surrounding the slit. The imager is capable of observing four wavelengths relating to emission at different altitudes as shown by Table \ref{iris-sj}.

\begin{table}[H]
\centering
\begin{tabular}{|c|c|c|c|c|}
SJI Passband & Wavelength \AA\ & FWHM \AA\ & Temperature $\log{T}$ & Region of Atmosphere\\
\hline
C II  & $1330$ & $40$ & $3.7 - 7.0$ & Upper-chromosphere\\
Si IV  & $1400$ & $40$ & $3.7 - 5.2$ & Upper-chromosphere\\
Mg II h/k & $2796$ & $4$ & $3.7 - 4.2$ & Lower-chromosphere\\
Mg II wing & $2832$ & $4$ & $3.7 - 3.8$ & Upper-photosphere\\
\end{tabular}
\caption{The IRIS/SJ is capable of observing four passbands, which relate to different plasma temperatures.}\label{iris-sj}
\end{table}


The IRIS spacecraft captured the temporal evolution of the flare between 14:09 and 17:54 UT via its slit-jaw imager and spectrograph at solar coordinates 518", 262", with a spatial resolution of 0.1667" per pixel. The slit-jaw imager data provides coverage of a field of view spanning 167" by 174", of passbands that including 1403, 2796 and 2832 \AA\ at 26, 19 and 75 second cadence respectively. The spectrograph slit has a field of view spanning 14" by 174" and is aligned directly over chromospheric flare ribbons, and the sunquake point of origin. For the majority of the observation, the spectrograph slit is exposed for $\sim9$ seconds at 8 slit locations for a total of 72 seconds cadence per raster. However, during the impulsive phase the IRIS SG shortens its exposure time to around 2.4 seconds in order to mitigate against saturation of the CCDs. Wavelengths observed over three channels include FUV1: 1331.7 - 1358.4 \AA, FUV2: 1389.0 - 1407.0 \AA\ and NUV: 2782.7 - 2851.1 \AA, associated with the transition region, chromosphere and the upper-photosphere. Spectral lines include C II, Si IV and Mg II h and k. IRIS SJI data is the standard level 2 data product provided for scientific research, which has been calibrated to negate dark currents, flat-field and spacecraft rotational effects. In order to observe flare ribbons in the photospheric data captured by IRIS SJI MG II wing channel, a running difference filter is applied. This effectively removes unwanted background features, highlighting the UV ribbons. IRIS SG data is manually corrected for changing exposure times and wavelength shifts caused by the orbital motions of the spacecraft. IRIS SG data is sampled over a wavelength range of 2825.7 and 2825.8\AA\ which represents a sample of Balmer pseudo-continuum. 

The next stage of the processing requires that IRIS SJ and SG data are converted from relative intensity (DN per pixel) to energy (erg) units. This is achieved by using a method provided in the instrument documentation \citep{2014SoPh..289.2733D} which calculates the conversion factors between relative DN s$^{-1}$ units and absolute erg s$^{-1}$ cm$^{-2}$ sr$^{-1}$ \AA\ $^{-1}$ units via equation \ref{irisradiometriccal}. Where $I_{dn}$ is relative intensity in units of DN per pixel, $C_{d2p}$ is the DN to photon conversion factor provided by \cite{2014SoPh..289.2733D}, $E_{\lambda}$ is the photon energy, $A_{eff}$ is the effective area, $t_{exp}$ is the exposure time, $d_{\lambda}$ is the wavelength dispersion, $\Omega$ is the solid angle and $I_{abs}$ is the intensity in absolute units.

\begin{equation}\label{irisradiometriccal}
I_{abs} = \frac{{I_{dn}} \; {C_{d2p}} \; {E_{\lambda}}}{{A_{eff}} \; {t_{exp}} \; {d_{\lambda}} \; {\Omega}} \text{ [erg s}^{-1}\text{ cm}^{-2}\text{sr}^{-1}\text{ \AA\ }^{-1}]
\end{equation}
%    fout = (array*n_pixels*dn2photon*E_photon)/(A_float*texp*pixlambda*w) ;erg/s.cm^2.sr.Å



\subsubsection{The Solar Dynamics Observatory's Helioseimic and Magnetic Imager}
Signatures of energy deposition in the lowest regions of atmosphere are captured by Solar Dynamics Observatory's (SDO) Helioseismic Imager (HMI), which is sensitive to the wing of the photospheric absorption line 6173 \AA\, which is essentially optical continuum. SDO/HMI has three main observables, continuum, dopplergrams and magnetic flux density, each of which can provide valuable insight into the physical conditions existing in photospheric plasma. In particular for this project, optical continuum data can provide information about WLFs intensity, which along with Balmer continuum could be linked to radiative backwarming of photospheric material, which is a possible sunquake progenitor. The point of origin and wave-fronts of the sunquake can also be detected using helioseismic data, which can be used to calculate acoustic power of the quake (see section \ref{helioholog}).
The HMI instrument on board SDO observes the entire solar photosphere with 4k x 4k CCD and a pixel size relating to 0.505" by 0.505", with each image having a cadence of 45 seconds. The data are calibrated to correct for cosmic-rays, dark currents, flat-field and spacecraft rotational effects. In this project, HMI continuum data is used primarily to observe white light flares, which are difficult to see against the bright photospheric background. To highlight the positions of flare ribbons in the photosphere, it is helpful to filter and remove other features from the data. Similarly to \cite{2014ApJ...783...98K}, photospheric continuum data captured by HMI is put through a two stage filtering procedure. First the data has an unsharp filter applied, so that the filtered image, $I_{filt}=I-$ smooth($I$,10), where $I$ is the original image and the function smooth relates to a 10 pixel boxcar smoothing filter, this is to remove small features such as granulation. The technique is not perfect meaning that some granulation is still visible after the unsharp filter. Second, $I_{filt}$ is subjected to a running difference filter to isolate locations that are white-light enhanced. The running difference filter effectively removes static features leaving behind those pixels that are changing over short time-scales. For the purpose of white light flare analysis, the data yield a strong contrast between flare-enhanced and background pixels after being filtered by a $i-2$ running difference. The next stage is to determine which pixels in the difference image are those that are enhanced during the flare. For the best result, white-light enhanced pixels are identified using a combination of visual inspection and thresholding. Attempts at automating the identification process tend to lead to false positives being triggered by noise or granulation features. \\
HMI Dopplergrams are used in conjunction with holography techniques to produce a 6mHZ acoustic egression power map (supplied via private communication by Sergei Zharkov) revealing the location of the sunquake. 

As with the IRIS data, the next stage of processing is to convert relative intensity into absolute intensity. To perform this conversion for SDO HMI data the basic equation \ref{irisradiometriccal} is used with values tailored for HMI. A combination of sources \citep{2012SoPh..275...41B, 2012SoPh..275..285C} have been used to find the instrument's properties which are used to fulfil a conversion factor. Where $g$ is the instrument gain, $QE$ is the quantum efficiency of the charged couple device and $r_{ap}$ is the instrument aperture radius, $\Upsilon$ is the relative filter assembly transmittance and all other terms are the same as equation \ref{irisradiometriccal}.


\begin{equation}\label{hmiradiometriccal}
I_{abs} = \frac{I_{dn} \; C_{d2p} \; E_{\lambda}}{A_{eff} \; t_{exp} \; d_{\lambda} \; \Omega} 
        = \frac{I_{dn} \; g \; QE \; E_{\lambda}}{\pi \; r_{ap}^{2} \; \Upsilon \; t_{exp} \; d_{\lambda} \; \Omega} \text{ [erg s}^{-1}\text{ cm}^{-2}\text{sr}^{-1}\text{ \AA\ }^{-1}]
\end{equation}


\subsection{Data Sampling}
For the IRIS SJI, IRIS SG and SDO HMI data sets, multiple sample points have been chosen based on the moment in time when the IRIS SG slit is directly over the southern flare ribbon and sunquake impact location. These coordinates are sampled across all data sets except RHESSI. Figure \ref{hmicontext} shows the sample coordinates, RHESSI HXR and the 6mHz sunquake egression map, plotted over filtered SDO HMI data in an effort to demonstrate the spatial alignment between HXR (particle beam), UV ribbons and sunquake impact. Table \ref{coordtab} shows how each sample number relates to a heliocentric position in arcseconds.


\begin{table}[h]
\centering
\begin{tabular}{|c|c|}
\hline
Sample Number & Heliocentric Position (x,y)\\
\hline
1 & 518.219", 262.000"\\
2 & 520.215", 263.000"\\
3 & 522.212", 262.000"\\
4 & 522.212", 265.000"\\
5 & 524.256", 265.000"\\
6 & 526.252", 263.818"\\
\hline
\end{tabular}
\caption{Sample number and the corresponding coordinates in heliocentric units (arcsec).}\label{coordtab}
\end{table}



