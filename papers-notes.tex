\documentclass[11pt]{article}
\usepackage{aas_macros}
\usepackage{hyperref}
\usepackage{appendix}
\usepackage[hmargin=1.5cm, vmargin=0.55cm]{geometry}
\usepackage{amsmath}
\usepackage{caption}
\usepackage{mathtools}
\usepackage{fancyhdr}
\usepackage{float}%Places the float at precisely the location in the LaTeX code...i.e, [H]
\usepackage{wrapfig}
\usepackage{rotating}
%\usepackage[pdftex]{graphicx}%includegraphics
%converts eps to pdf
%\usepackage[update,prepend]{epstopdf}
%\usepackage{subcaption} %need to find subcaption.sty for this to work!
%\usepackage{epstopdf}

%\graphicspath{ {~/PhD/Thesis/upgrade-plots/} {./} }
\DeclareGraphicsExtensions{.pdf,.png,.jpg,.eps,.gif,.ps}
\renewcommand{\vec}[1]{\mathbf{#1}}
\usepackage[round]{natbib}  %use the "Natbib" style for the references in the Bibliography
%\usepackage{aastex}%defines journal abbreviations in bib file
%\newcommand{name}[num]{definition}



\title{A collection of Notes from Important Papers Relating to Sunquakes and Solar Flares.}
\author{Jamie Ryan \\
Mullard Space Science Laboratory \\
University College London \\
Surrey, RH13 6NL, UK\\
\href{mailto:jamie.ryan.14@ucl.ac.uk}{jamie.ryan.14@ucl.ac.uk}
\date{}}
\begin{document}
\maketitle
\tableofcontents

%%%%%%%%%%%%%%%%%%%%%%%%%%%%%%%%%%%%%%%%%%%%%%%%%%%%%%%%%%%%%%%%%%%%%%%%%%%%%%%%%%%%%%%%%%%%
\section{Prelude}
This document contains a collection of notes taken whilst reading various papers. Each section relates to one specific paper, although there maybe references to other papers included within said section. Any notes that are \emph{emphasized} are my own thoughts, and are not statements that are contained in the paper. All notes are written in my own words, therefore, all notes can be used as text bodies for my own papers or thesis write up.


\section{Hydrogen Balmer Continuum In Solar Flares Detected By The Interface Region Imaging Spectrometer (IRIS)}
The following notes are taken from: \cite{2014ApJ...794L..23H}. \\

\textbf{On page 1, in the introduction: } As described by the collisional thick target model (CTTM \cite{1971SoPh...18..489B}), electrons accelerated by the reconnecting coronal magnetic field, penetrate the into the lower atmosphere depositing energy along the way. Emission in the lower atmosphere is due to; heating, causing various line and continua emission; and collisions by non-thermal electrons exciting and ionizing the local plasma. If emission is of a wavelength comparable to the visible spectrum then the term white light flare is used to describe it. \emph{That being said, white light flares also emit in wavelengths at the extremes of the visible range, such as NUV.} Continua emission contributing to white light flares are thought to occur via two processes, heating of the temperature minimum region, and hydrogen recombination at chromospheric altitudes \cite{2007ASPC..368..417D}. Downward directed hydrogen recombination continuum emission is associated with radiative backwarming of the photosphere, with the upward component being observable. The same hydrogen population also emits in EUV continuum below 912\AA\ due to atomic Lyman transitions, emission which has been observed recently using the SDO/EVE instrument \citep{2012ApJ...748L..14M, 2014ApJ...793...70M}. \emph{searching for Milligan et al papers I stumbled across a paper that attempts to constrain plasma densities during a sample of X class flares. Might be very useful as a reference to draw upon for my own density approximations.} An estimate of the radiative energy in the optical has recently been made by \cite{2013ApJ...776..123W, 2014ApJ...783...98K} and \cite{2014ApJ...793...70M} using Hinode/SOT. The range of the spectrum covered by these estimates is small, and data is converted to energy units via fitting to a blackbody (BB) curve. This approach is not ideal, as the BB predicts low levels of emission in the Balmer continuum which contradicts increased levels produced using numerical simulations of the hydrogen recombination process. \emph{There is no reference for the paper containing these hydrogen recombination simulations???? I found a nice paper by Adam Kowalski \cite{2015SoPh..tmp...61K}, which may contain such simulations. May be it's worth emailing Adam to see if he wants to collaborate by running simulations based on my observations. Also, \cite{2014ApJ...793...70M} may contain a reference to such simulations.}. Observations of Balmer emission are desirable in order to determine the accuracy of such simulations and constrain models of WLF production. Most Balmer observations have been made form ground based telescopes at around the Balmer-limit of 3646 \AA\. Some of this work detected the Balmer jump, whilst others observed a smooth transition from blue \citep{1985A&A...152..165D} to Balmer continuum \citep{1989SoPh..121..261N}. Observations contained in this paper are novel due to the fact that they are of a part of the spectrum which is beyond the Balmer limit. This helps to eliminate some of the usual difficulties in observing WLFs because flare contrast at this spectral range is greater than at visible wavelengths.\\
\textbf{On page 2, in the Observations: } S$\mu = \cos\theta = 0.83$. \emph{This page has a nice figure showing locations of Balmer continuum as orange lines or blocks.} \\
\textbf{On page 3, in the Observations: } NUV channel provided by the IRIS spectrograph is technically capable of observing a wavelength range of 2783 to 2835 \AA\, however, due to downlink constraints and to save time, only some of the is provided. This particular data set contains the "flare linelist" spectra which includes 2791 to 2806 \AA\, 2813 to 2816 \AA\, 2825 to 2828 \AA\ and 2831 to 2834 \AA\ wavelength windows, and is the level 2 science product. \emph{level 2 includes dark current, flatfield and geometric calibrations}.\\
\textbf{On page 3, in the Analysis of IRIS Flare Spectra: } IRIS data is analysed for continuum enhancement during the flare at the far Mg II wing part of the spectrum which includes no visible spectral lines, sometimes referred to as quasi-continuum and was also observed by the HRTS-9 mission \citep{2008ApJ...687..646M}. Horizontal lines in the IRIS spectra intensity plots are where continuum increases by at least 30 DN/s when compared to average counts. Count increase is calculated by looking at each slit position and subtracting an average, e.g., at slit position one: 17:46:51 - avg(17:40:36 - 17:44:21) [DN/s]. Timings are different for each slit position due to the time taken for each raster. These horizontal enhancements spanning the spectrum coincide with ribbon locations. Plots of the spectrum at different y-axis pixel locations show that continuum counts are enhanced by two to three times that of the "quiet" sun. \emph{the paper uses a bunch of different sample points for spectra}; compared to a set of reference spectra, lines contained within the ribbons turn into emission-lines during the flare, and there is also an increase in continuum; areas with enhancement outside of the flare region have a similar continuum increase, but a lack of emission lines, showing that continuum inbetween emission-lines is not affected by the emission. \\
\textbf{On page 4, in the Analysis of IRIS Flare Spectra: } There is a plot of spectra from various y-pixel locations, y = 423 (max redshift \emph{....south ribbon?}), y = 447 (bright continuum \emph{....north ribbon?}), y = 468 (bright upper strip \emph{....north ribbon?}), y = 620 (quiet sun), y = 314 (bright outside flare region). Looking at the behaviour of the continuum enhancement over time can show whether the increased is due to the flare, so lightcurves are made for each spectral sample. The data is averaged over the continuum range 2825.7 to 2825.8 \AA\ and plotted (DN vs time) each time the slit is in the same position. Lightcurves for y = 423 and y = 447 show the most obvious impulsive features coinciding with RHESSI flare data. y = 468 shows a slight increase at the impulsive phase. y = 314 and y = 620 both show no activity associated with the flare. \emph{Next, the author plots another spectrum with QS values subtracted to gain the absolute enhancement increase (difference spectra)}; continuum absolute increase is around 20DN/s before and 80DN/s during the flare. All spectral regions show an increase in their repective continuum regions, so there is no wavelength dependance on continuum brightness. Difference spectra also show that spectral lines vary more than the continuum, although they show a constant in that there is no brighteneing of the far h-line wing during the flare. \\
\textbf{On page 4, in the Hydrogen Recombination Continuum: } If the balmer continuum is generated by hydrogen recombination then observational data can be compared to theoretical models. This requires a conversion from DN/s to erg.$sec^{-1}cm^{-2}sr^{-1}\AA^{-1}$ (cgs units). \emph{Now Balmer simulations are referenced from \cite{2007ASPC..368..417D}...printed out ready to read!}. The observations at y = 620 of the quiet sun show approximately 50DN/s which using the results from HRTS-9 is calculated to be $3.7{\times}10^{5}$ in cgs units. y = 447 has a pre-flare value of 60DN/s and a flare value of 115DN/s, therefore subtracting the pre-flare gives a flare enhancement of 55DN/s corresponding to $4.1{\times}10^{5}$ cgs units. Work by various authors with NLTE models of Balmer recombination are documented in \cite{2007ASPC..368..417D}. \\
\textbf{On page 5, in the Hydrogen Recombination Continuum: } This paper uses static flare models from \cite{1983ApJ...272..739R} which include a grid of models in energy balance with the electron beam (CTTM) and conductive energy deposit in the lower atmospheric layers. Using their model E4 with electron-beam flux $F_{20}$ equal to $10^{11}$ erg.$sec^{-1}cm^{-2}$ with a spectral index of $\delta = 5$ \emph{Same $\delta$ as in RHESSI fitting, i.e, cttm}. Using MALI (Multilevel Accelerated Lambda Iteration) NLTE technique, combined with non-thermal collisional rates for hydrogen, intensity of recombination Balmer continuum is calculated for the specific model. The model predicts $3.2{\times}10^{5}$ cgs units with the region of the chromosphere having an optical thickness of 0.1 (i.e, it is an optically thin region as expected). \emph{why use the CTTM model if the region is optically thin???}. Therefore, the value calculated from the observations ($4.1{\times}10^{5}$ cgs units) is consistent with the model prediction. In terms of the model, increasing the pressure of plasma within coronal loops increases Balmer continuum enhancement within the emitting layer. The calculated value is also in agreeance with the semi-empirical model (F2) put forward by \cite{1986lasf.conf..216A} which predicts a similar Balmer continuum, however, for very large flares (F3) they predict a much greater (by a factor of 3) Balmer continuum. In emitting optically thin layers, the Balmer continuum flare contrast increases substantially as \mu diminishes, thus for observations close to the limb, it should be easier to detect Balmer continuum.  \\
\textbf{On page 6, in the Hydrogen Recombination Continuum: } IRIS radiometric calibration is tested against HRTS-9, with the quiet sun value being around three times larger. IRIS rad cal error is supposed to be maximum a fcator of two, while hrts-9 should also have similar error.
\textbf{On page 6, Discussion and Future Prospects: }
There is strong Balmer continuum at locations of strong Mg II line emission. There are areas of increased Balmer continuum outside of the flare region. This is probably due to variations in photospheric continuum in this UV spectral range even for the quiet-sun. 

\textbf{Final Summary In My Words.}
\begin{itemize}
\item Balmer continuum enhancement generated by the flare seems to align spatially with Mg II line emission.
\item Balmer continuum emission exists outside of the flare region, and is probably due to variations in the UV range of photospheric continuum. Even for the quiet sun.
\item Balmer continuum enhancements at flare locations exceed enhancements outside of the flare region significantly.
\item Time-series data of Balmer continuum follows the evolution of the flare, and within ribbon locations increases impulsively, which is consistent in time with the RHESSI lightcurve.
\item The impulsive increase and gradual decrease look similar to findings by \cite{2014ApJ...783...98K} (Fig 5 and 6)
\item The work by Kerr and Fletcher (referenced above) considers an optically thin slab producing hydrogen recombination emission, the same process was considered for interpreting the data in this paper.
\item Further RHD simulations are required to predict Balmer and Paschen continua lightcurves consistently with energy deposition by an energetic particle beam.
\item In the far-wing of the Mg II h line, Balmer continuum enhancement is added on top of the quiet-sun spectrum. (NEED TO UNDERSTAND THIS.)
\item    
\end{itemize}

\label{Bibliography}
\lhead{\emph{Bibliography}}  % Change the left side page header to "Bibliography"
%\bibliographystyle{unsrtnat}  % Use the "unsrtnat" BibTeX style for formatting the Bibliography
\bibliographystyle{plainnat}%abbrv}
\bibliography{../Bibliography/Bibliography}  % The references (bibliography) information are stored in the file named "Bibliography.bib"
%\bibliography{Bibliography}
\end{document}
