%%%%%%%%%%%Observable Seismic Signatures%%%%%%%%%%%%%%%%%%%%%%%%%%
\subsection{Observable Seismic Signatures}
%Basics of helioseismology and challenges of observing acoustic emission
\subsubsection{Local Helioseismology}
%use content from old report...maybe expand a little






Helioseismology is a tool for probing the interior of the Sun. Most techniques in this field of analysis rely on observations of gravity and acoustic waves on the photosphere that are the result of interior excitation. Studying the frequency and modes of these oscillations has revealed much about the internal structure of the Sun. Local helioseismology then is a collection of techniques developed for global helioseismology have been modified for use in studying local regions in higher spatial resolution. The following section provides a very basic introduction to some of these techniques. 


\paragraph{Helioseismic Holography}\label{helioholog}
Originally the idea of analysing Doppler images of the solar surface in order to observe acoustic sources was put forward by \cite{1975CRASB.281...93R}. Helioseismic holography was developed further in concept by Lindsey and Braun \citep{1990SoPh..126..101L, 1992ApJ...392..739B, 1997ApJ...485..895L} in an effort to to image the solar interior and far-side of the Sun. This technique involves using a Doppler image as a representation of the wave-field at a location on the solar surface as a reference point to be able to estimate that wave-field a location in the solar interior at a time preceding or proceeding the image. This is achieved by calculating the ingression or egression of the wave-field by assuming that it's evolution is a, convergence to, or divergence from, the point of origin of that wave-field.  



\paragraph{Time-Distance}\label{TD}
The paper by \cite{1993Natur.362..430D} explains how to extract time-distance (TD) information from observations of intensity fluctuations on the solar surface. This technique uses travel times of waves between two locations on the solar surface. The method assumes that the travel time of a wave propagating in the interior of the Sun will be modified by any anomalies that it has to travel through, thus the resulting signal will contain the signatures of those irregularities. For instance, if the wave encounters a flow along it's path of travel, it will propagate faster with the flow than against it, affecting travel time.     

