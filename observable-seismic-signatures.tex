%%%%%%%%%%%Observable Seismic Signatures%%%%%%%%%%%%%%%%%%%%%%%%%%
\subsection{Observable Seismic Signatures}
%Basics of helioseismology and challenges of observing acoustic emission
Helioseismology is a tool for probing the interior of the Sun. Most techniques in this field of analysis rely on observations of gravity and acoustic waves on the photosphere that are the result of interior excitation. Studying the frequency and modes of these oscillations has revealed much about the internal structure of the Sun. Local helioseismology is a collection of techniques developed for global helioseismology that have been modified for use in studying local regions in higher spatial resolution. The following section provides a very basic introduction to some of these techniques. 

\subsubsection{Local Helioseismology}
%use content from old report...maybe expand a little

\paragraph{Helioseismic Holography}\label{helioholog}
\cite{1999ApJ...513L.143D} pioneered the use of helioseismic holography to produce seismic images of the solar flare of July 1996 reported to have a sunquake by Kosovichev and Zharkova. Time series egression-power maps at 3.5 and 6 mHz were computed with a 2 mHz bandwidth. It was found that the most powerful acoustic power frequency associated with the flare is centred at 3.5 mHz but has a large signal to noise ratio. Whereas the 6 mHz range has a much lower ambient noise, therefore producing a better rendering of the seismicity of the flare. It is now standard practice to use the 6 mHz range for helioseismic holographic calculations of egression-power. \\
Originally the idea of analysing Doppler images of the solar surface in order to observe acoustic sources was put forward by \cite{1975CRASB.281...93R}. Helioseismic holography was developed further in concept by Lindsey and Braun \citep{1990SoPh..126..101L, 1992ApJ...392..739B, 1997ApJ...485..895L} in an effort to to image the solar interior and far-side of the Sun. This technique involves using a Doppler image as a representation of the wave-field at a location on the solar surface as a reference point to be able to estimate that wave-field a location in the solar interior at a time preceding or proceeding the image. This is achieved by calculating the ingression or egression of the wave-field by assuming that it's evolution is a, convergence to, or divergence from, the point of origin of that wave-field. 
This technique uses Green's function (eqn \ref{green}, where $\vec{r}$ and $t$ are position and time of an observed signal and $\vec{r}'$ and $t$' are the position and time of the signal earlier in time) which assumes the that the acoustic wave propagates from a point source, allowing a signal $\psi(\vec{r},t)$ observed on the surface to be devolved backwards in time. 

\begin{equation}\label{green}
G_{+}(|\vec{r}-\vec{r}'|,t-t')
\end{equation}

Where $a$ and $b$ constrain the holographic pupil, equation \ref{holog} is then used to devolve the surface signal to calculate the position of subsurface acoustic sources. 

\begin{equation}\label{holog}
H_{+}(\vec{r},z,t)= \int dt'  \int_{a<|\vec{r}-\vec{r}'|<b} d^{2}\vec{r}'G_{+}(|\vec{r}-\vec{r}'|,t-t')\psi(\vec{r}',t')
\end{equation}

Equation \ref{eggpower} is then used to calculate the egression power associated with the acoustic sources at a time $t$. 

\begin{equation}\label{eggpower}
P(z,\vec{r})=\intdt|H_{+}(\vec{r},z,t)|^{2}dt
\end{equation}

If egression power is required in terms of frequency then equation \ref{eggpower} can be Fourier transformed into frequency space.


\paragraph{Time-Distance}\label{TD}
The first observation of a sunquake \citep{1998Natur.393..317K} used the time-distance technique to track sunquake wavefronts. The paper by \cite{1993Natur.362..430D} explains how to extract time-distance (TD) information from observations of intensity fluctuations on the solar surface. This technique uses travel times of waves between two locations on the solar surface. The method assumes that the travel time of a wave propagating in the interior of the Sun will be modified by any anomalies that it has to travel through, thus the resulting signal will contain the signatures of those irregularities. For instance, if the wave encounters a flow along it's path of travel, it will propagate faster with the flow than against it, affecting travel time.     
This technique remaps Dopplergrams into polar coordinates, with the point of origin centered on the area of downflowing material during the flare. This remapped image is then Fourier transformed with respect to azimuthal angle, with the resulting image highlighting cicular disturbances as a line of positive slope.    
