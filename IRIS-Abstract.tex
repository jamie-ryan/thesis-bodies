\documentclass[10pt]{article}
\usepackage{hyperref}
\usepackage[hmargin=1.25cm, vmargin=0.55cm]{geometry}
\begin{document}
\title{IRIS-4 Abstract}
\author{Jamie Ryan \\ 
Mullard Space Science Laboratory \\
University College London \\
Surrey, RH13 6NL, UK\\
\href{mailto:jamie.ryan.14@ucl.ac.uk}{jamie.ryan.14@ucl.ac.uk} \\
\date{\today}
}
\maketitle




Local helioseismic events or sunquakes are the propagation of acoustic waves in the subphotosphere in response to solar flares. The progenitors of sunquakes are thought to be either shocks, radiative backwarming, direct particle collision or sudden magnetic field reconfiguration. Each of these mechanisms relies on the transport of energy from the corona to the photosphere, and the physical parameters existing in the chromosphere such as magnetic configuration and density. Thus, to understand sunquakes and their relationship to solar flares, we need to understand how energy moves through the entire solar atmosphere and the physical parameters that are present. White light flares are often present at the sunquake's point of origin, and although they are difficult to observe due to the bright background optical emission, they can provide insight into mechanisms of energy transport to the photosphere. The majority of the energy released by a flare is deposited in the lower solar atmosphere and manifests itself in the form of enhanced hard X-ray, UV and optical radiation. Hard X-ray footpoints and UV ribbons observed in the chromosphere directly map to the reconfiguring magnetic fields during the flare. Ribbons therefore provide an important observational tool for understanding the magnetic reconnection process and how magnetic energy is transported from high to low altitudes. Using observations of an X1 solar flare with associated seismicity on the 29th of March 2014 from the Solar Dynamics Obervatory's (SDO) Helioseismic Imager (HMI) and Interface Region Imaging Spectrograph (IRIS), white light-enhanced regions in the photosphere are spatially and temporally aligned with higher altitude ribbons in the chromosphere. Energies associated with the flare at various altitudes are calculated to constrain when and where energy is being deposited in the solar atmosphere. Spectroscopic data from IRIS is used to determine temperatures and densities in the chromosphere, providing information needed to identify which of the various progenitors of sunquakes are present.         




\end{document}
