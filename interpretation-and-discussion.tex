\section{Interpretation and Discussion}
%NEW%%%%%%%%%%%%%%%%%%%%%%%%%%%%%%%%%%%%%%%%%%%%%%%%%%%

%RHESSI
The spatial and temporal alignment of 10 t0 100 keV HXR and emission from lower altitudes (Figure \ref{}) is further backed up by \cite{2016ApJ...816...88K}. They show that peak emission during the flare from various continua are in good alignment in space and time with the 30 to 100 keV HXR source.  

\cite{1998Natur.393..317K} suggest that the stimulation of a sunquake by direct particle beam interaction is caused by an injection of electrons aligned with the acoustic impact location. For this reason, the electron beam in the 29th of March flare is a candidate progenitor of the observed sunquake. 

However, further analysis has shown that the momentum carried the electron beam is insufficient to generate the quake. Furthermore the derived proton beam momentum also falls short, meaning that the sunquake is not generated by particle beam collision.    



$p_{sqk}$ is at least three orders of magnitude higher than electron and proton beam momenta $p_e$ and $p_p$. This means that direct particle collision is unlikely to be causing the sunquake. 


%Balmer continuum enhancement is in angreeance with 1D... models by \cite{}

%HMI shows an enhancement of ... which is similar to values claculated by \cite{}. This means that the novel energy conversion method employed in the data reduction produces a reliable result.



So what is causing the sunquake?

shocks? 
\cite{2015ApJ...812...35M} show that there is substantial redshifted spectral lines during the impulsive phase that indicate the existence of shocks. 


Dissapation of Alfven waves?
\cite{1982SoPh...80...99E} mentioned by \cite{2015ApJ...812...35M} as a possible energy transport mechanism maybe capable of producing a sunquake.

Lorentz force?
\cite{2014ApJ...796...85J} show that the Lorentz force is not capable of producing the sunquake  

