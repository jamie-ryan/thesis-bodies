\section{Interpretation and Discussion}
%NEW%%%%%%%%%%%%%%%%%%%%%%%%%%%%%%%%%%%%%%%%%%%%%%%%%%%
The momentum needed to produce the sunquake is

\begin{equation}
p_{sqk}\sim \rho \; l^{3} \; v
\end{equation}\label{sqk-momentum} 

where all terms are tailored for photospheric values, hence density $\rho \sim 10^{-8}$g cm$^{-3}$; sound speed $v \sim 10^{6}$ cm.s$^{-1}$ \citep{2015ApJ...807..102S}. The length-scale, $l \sim  1.82{\times}10^{8}$, corresponds to the sunquake impact diameter, 

\begin{equation}
l = 2\sqrt{\frac{A_{sqk}}{\pi}}
\end{equation}\label{lengthscale}

%Equation \ref{sqk-momentum} gives the momentum of the acoustic reponse as $p_{sqk} = 6.03{\times}10^{22}$ g cm s$_{-1}$, which when compared to particle beam momenta, $p_e$ and $p_p$, is $10{5}$ and $10^{4}$ times larger respectively. 

An alternative method of calculating the momentum associated with the sunquake is to use the $P_{sqk}$. By further developement of equation \ref{sqk-momentum} shown in the Appendices section \ref{sqk-ptoP-relation}, sunquake momentum becomes,  

\begin{equation}
p_{sqk} \; \sim \; \frac{P_{sqk} \; \rho \; v \; l }{F_{sqk}}
\end{equation}\label{sqk-momentum-from-power}

where acoustic emission flux, $F_{sqk} = $ (see equation \ref{Psqk}).

% equation \ref{sqk-momentum-from-power} produces a sunquake momentum of $p_{sqk} \sim 4.73{\times}10^{22}$ g cm s$^{-1}$.

A range of velocities can be input into equations,\ref{sqk-momentum} and \ref{sqk-momentum-from-power}, allowing the consideration of lower and upper limits of acoustic impact momentum. The momentum values resulting from inputing sunquake wavefront velocity of $v_{sqk} = 8.0{\times}10^{8}$ cm s$^{-1}$ \citep{2014ApJ...796...85J} and the photospheric sound speed into equations \ref{sqk-momentum} and \ref{sqk-momentum-from-power} are shown in Table \ref{sqk-momenta}


\begin{table}[h]
%\tiny
\centering
\begin{tabular}{|c|c|c|}
Velocity $v$ [cm s$^{-1}$] & Eqn \ref{sqk-momentum} $p_{sqk}$ [g cm s$^{-1}$] & $ Eqn \ref{sqk-momentum-from-power} $p_{sqk}$ [g cm s$^{-1}$]\\
\hline
$8.0{\times}10^{8}$ (sunquake wave speed) & $4.82{\times}10^{25}$ & $3.79{\times}10^{25}$\\
$1.0{\times}10^{6}$ (photospheric sound speed) & $6.03{\times}10^{22}$ & $4.73{\times}10^{22}$\\
\end{tabular}
\caption{Acoustic impact momenta, caclulated using by inputting values shown in the 'Velocity' column into equations \ref{sqk-momentum} and \ref{sqk-momentum-from-power}. These values provide an upper and lower limit on sunquake impact momentum.}\label{sqk-momenta}
\end{table}

Taking upper and lower limits for calculated acoustic source momentum as $4.73{\times}10^{22} \leq p_{sqk} \geq 4.82{\times}10^{25}$ g cm s$^{-1}$, comparisons can be made with particle beam and radiation pressure momenta.

\subsection{Partcle Beam}
Using the nonthermal power, $P_{e}$ from the RHESSI HXR data fit, detailed in Section \ref{rhessiresults}, the flux input by the nonthermal particle beam can be calculated as,

\begin{equation}\label{electronflux}
F_e = \frac{P_{e}}{A_{HXR}}
\end{equation}

Flux $F_e$ is in units of erg s$^{-1}$ cm$^{-2}$ and the HXR emission area is determined by the $90\%$ HXR contour in Figure \ref{hmicontext}, rendering $A_{HXR} \sim \pi (2"{\times}7.25{\times}10^{7})^{2} = 6.61{\times}10^{16}$ cm$^{2}$. The resulting flux turns out to be $F_e \sim 10^{44}$ erg s$^{-1}$ cm$^{-2}$ which when multiplied by the sunquake area, can provide an upper limit for the nonthermal power available for generating an acoustic disturbance $P_{e}(A_{sqk}) = \frac{F_e}{A_{sqk}}$. So using the available nonthermal energy, $P_e \sim 2.5{\times}10^{29}$ erg.s$^{-1}$ and $A_{sqk} \sim 2.6{\times}10^{16}$ $cm^{2}$. $P_{e}(A_{sqk}) \sim 10^{28}$ erg s$^{-1}$ and when compared with the sunquake power $P_{sqk} \sim 1.3\pm0.05{\times}10^{26}$ $erg.s^{-1}$ it is clear that the electron beam does have the required 1000 times more energy required to the sunquake. Another interesting quantity to investigate is the momentum of the particle beam. Electron momentum can be calculated by 


\begin{equation}\label{electron-momentum}
p_e=\tau \sqrt{2m_e} P_{e}
\end{equation}

where $m_e$ is electron mass, $\tau$ is the time duration of flare impulsive phase and $P_{e}$ is described by equation \ref{pnth1} \citep{2015ApJ...807..102S}. Substituting values in to equation \ref{electron-momentum} yields an electron momentum of $p_e \sim 1.35{\times}10^{17}$ g cm s$^{-1}$. Assuming the energy in the electron beam is equal to that in a population of accelerated protons \citep{2000ApJ...542..513E}, then calculation of a theoretical proton beam momentum, where $m_p$ is the proton mass, is by the relation,

\begin{equation}\label{proton-momentum}
p_p \sim p_e \sqrt{\frac{m_p}{m_e}}
\end{equation}

Which yields a proton momentum of $p_p \sim 5.79{\times}10^18$ g cm s$^{-1}$, an order of magnitude greater than $p_e$. This is because $m_p ~ 2000m_e$ meaning the square root in equation \ref{proton-momentum} renders the result $p_p \sim 44.7p_e$. 

Comparing $p_{e}$ and $p_{p}$ with the lower limit of $p_{sqk}$, the electron and proton beam carry $\sim 10^{5}$ and $\sim 10^{4}$ times less momentum than $p_sqk$ respectively. This means that even if an electron or proton beam can make it down to the photosphere, it wouldn't have the necessary momentum to cause the sunquake on it's own. In reality, calculated momenta are idealistic, not treating the effects of energy loss due to energy dissapation. The point being that for a particle beam to make it to the photosphere, it has to traverse 9 pressure scale heights, increasing in density with depth. As density increases particles in the beam are more likely to encounter ambient plasma, and as a result are deccelerated, giving up as energy as emission. So if or when the beam reaches the photosphere, much of it's energy and momentum has already been dissapated in the chromosphere, making the generation of a sunquake via just the particle beam even more unlikely. \\

However, energy deposited into the atmosphere by the particle beam can lead to other sunquake generation mechanisms, such as radiative backwarming and shocks which are described in sections \ref{sunprog}. A recent result from \cite{2016ApJ...816...88K} shows that the intensity of Balmer continuum observed in this event could come from either; 23\% of nonthermal electrons with energy $<20$ keV; or the entire population of nomthermal electrons with energy $<40$ keV. This means that a large portion of energy delivered to the lower atmosphere by the electron beam is dissapated causing various continua and emission lines. \\

The flux values for Balmer and HMI continuum shown in Figure \ref{fluxladder-balm-hmi-only} can be used to estimate the power of the radiative backwarming. The key being whether the radiative backwarming is powerful enough to generate the white light flare and hence the sunquake. The power profiles shown in Figure \ref{powerladder-balm-hmi-only} are calculated by assuming a homogenous energy distribution in the region surrounding each coordinate, it is then possible to use the relation,

\begin{equation}
P_{Balm} = F_{Balm} \; A_{sqk}  
\end{equation}\label{Pbalm}

where $P_{Balm}$ is the power of the Balmer continuum emitted from an area equal to the sunquake, $A_{sqk}$. The same data set and coordinate scheme is followed as in Figure \ref{fluxladder-balm-hmi-only}. The Balmer continuum over the sunquake shows an impulsive power $P_{Balm} = 6{\times}10^{13}$ erg.s$^{-1}$ which is thirteen orders of magnitude smaller than the power of the sunquake $P_{sqk} \sim 1.3\pm0.05{\times}10^{26}$ $erg.s^{-1}$. This means that there is not enough energy per second deposited by radiative backwarming to create the sunquake. The HMI continuum power, $P_{HMI}$, over the sunquake location peaks at $P_{HMI} = 2{\times}10^{14}$ erg.s$^{-1}$cm$^{-2}$, which is twelve orders of magnitude less than the $P_{sqk}$ but ten times greater than the Balmer continuum. One of the biproducts of radiative backwarming are white light flares in the photosphere. Balmer continuum radiated outward to the observer, is supposed to be equal in power to that emitted downward \citep{1989SoPh..124..303M}. In that case the white light flare shown in the HMI continuum in Figure \ref{powerladder-balm-hmi-only} is only provided $10\%$ of its power by radiatve backwarming. So radiative backwarming at first glance may not be causing the observed white light emission. However, another way to investigate the energy deposition in the lower atmosphere is by integrating radiative flux over the impulsive phase of the flare. This provides an upper limit for the total flux injected into the system during the impulsive phase, which can be used to calculate the total emission power. The integrated flux is calculated,

\begin{equation}
F_{imp} = \int_{0}^{\tau} F(t) \; dt = F(t) \; \tau
\end{equation}\label{f-imp}
 
where the duration of the impulsive phase $\tau = $ and $F(t)$ is the emitted flux at time $t$. The total power emitted during the impulsive phase, $F_{imp}$ is

\begin{equation}
P_{imp}=F_{imp} \; A_{sqk}
\end{equation}\label{e-imp}

where it is assumed that a homogenous energy distribution exists throughout the sunquake impact area. This produces values for each data set tabulated in Table \ref{eimp}. Balmer and HMI continua are the data sets that show the highest integrated energy levels, with comparable values at each coordinate. The fact that Balmer and HMI continua show such similar energies emitted over the impulsive phase means that radiative backwarming is likely causing the white light flare in the photosphere as described in section \ref{wlf}. The highest energy reading in the HMI continuum is over the sunquake, with a value of $1.42{\times}10^{16}$ erg whereas in the Balmer continuum the highest is coordinate three with $2.52{\times}10^{16}$, both of which are ten orders of magnitude smaller than $P_{sqk}$. Meaning that the radiative backwarming mechanism in this case is not powerful enough to produce the sunquake.    








%RHESSI
The spatial and temporal alignment of 10 t0 100 keV HXR and emission from lower altitudes (Figure \ref{}) is further backed up by \cite{2016ApJ...816...88K}. They show that peak emission during the flare from various continua are in good alignment in space and time with the 30 to 100 keV HXR source.  

\cite{1998Natur.393..317K} suggest that the stimulation of a sunquake by direct particle beam interaction is caused by an injection of electrons aligned with the acoustic impact location. For this reason, the electron beam in the 29th of March flare is a candidate progenitor of the observed sunquake. 

However, further analysis has shown that the momentum carried the electron beam is insufficient to generate the quake. Furthermore the derived proton beam momentum also falls short, meaning that the sunquake is not generated by particle beam collision.    



$p_{sqk}$ is at least three orders of magnitude higher than electron and proton beam momenta $p_e$ and $p_p$. This means that direct particle collision is unlikely to be causing the sunquake. 


%Balmer continuum enhancement is in angreeance with 1D... models by \cite{}

%HMI shows an enhancement of ... which is similar to values claculated by \cite{}. This means that the novel energy conversion method employed in the data reduction produces a reliable result.



So what is causing the sunquake?

shocks? 
\cite{2015ApJ...812...35M} show that there is substantial redshifted spectral lines during the impulsive phase that indicate the possible existence of shocks. 


Dissapation of Alfven waves?
\cite{1982SoPh...80...99E} mentioned by \cite{2015ApJ...812...35M} as a possible energy transport mechanism maybe capable of producing a sunquake.

Lorentz force?
\cite{2014ApJ...796...85J} show that the Lorentz force is not capable of producing the sunquake  

