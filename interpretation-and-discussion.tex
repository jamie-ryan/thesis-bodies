\subsection{Interpretation and Discussion}
This project presents a first look at a multi-wavelength energy analysis of the lower solar atmosphere during the 29th of March 2014 X class solar flare. The main focus up to this point has been to acquire energy estimates from the various atmospheric regions in an attempt to assess the likelihood of radiative backwarming as a generation method for the sunquake. The location of maximum acoustic power, RHESSI HXR, IRIS and SDO intensity correlate both spatially and temporally (see Figure \ref{saxcontours-vert}), showing that energy input into the upper chromosphere via accelerated non-thermal electrons propagates down to the photosphere. The sunquake is located directly underneath both maximum HXR and an area of white-light emission.

Energy estimates from RHESSI data show there to be between $1.0{\times}10^{28}$ to $2.5{\times}10^{29}$ erg during the impulsive phase of the flare. Comparing this to the sunquake energy of $1.3\pm0.05{\times}10^{26}$ $erg.s^{-1}$ means that the acoustic energy is well within the energy budget provided by accelerated non-thermal electrons, so how is the energy getting to the photosphere to cause seismic event?

Energy estimates from IRIS are not always useful due to over saturation of the instrument CCD, however, when comparing the ribbon locations to that of the sunquake some clear behaviours reveal themselves. The 1400 \AA\ channel is almost always showing greater radiative energy in all ribbon locations compared to the sunquake. This could be because transition region ribbons are rarely directly above the sunquake epicentre due to magnetic field configuration. The 2796 \AA\ channel becomes saturated slightly less often than the 1400 \AA\, providing more insight. The IRIS 2796 \AA\ channel is less in energy than the sunquake location around 30\% of the time. However, no real comparison can be attained due to the saturation of the CCD at the majority of ribbon pixels and the sunquake pixel. The only information that can be deduced, is that the sunquake pixel has an extraordinary 2796 \AA\ enhancement but then so do the majority of the ribbon locations. Balmer emission calculated from IRIS SG data is consistently of less energy than at the sunquake location, this is an interesting result as it means that there is more radiative backwarming above the sunquake pixel than in any other location sampled. However, the energy required to generate the sunquake is in the order of $10^25$ ergs, which when compared to the $2.68{\times}10^{16}$ erg of energy available in the Balmer pixel is $10^{9}$ times greater! Perhaps radiative backwarming only plays a bit-part in the generation of the sunquake?  
The 2832 \AA\ IRIS channel is harder to summarise due to the fact that comparing ribbon locations to that of the sunquake show that there is a 50:50 chance that the ribbon could have greater or lower energy output.
SDO HMI continuum data is probably the most revealing, in that almost all of the ribbon locations have far greater energy output than the sunquake location. This could be explained by the sunquake being only partially under a white light ribbon. 

$p_{sqk}$ is at least three orders of magnitude higher than electron and proton beam momenta $p_e$ and $p_p$. This means that direct particle collision is unlikely to be causing the sunquake. 
