%%%%%%%%%%%%%%%%%%%%%%%%%%%%%%%%%%%%%%%%%%%%%%%%%%%%%%%%%%%%%%%%%%%%%%%%%%%%%%%%%%%%%%%%%
%%%%%%%%%%%Progenitors of sunquakes%%%%%%%%%%%%%%%%%%%%%%%
\subsubsection{Sunquake Progenitors}\label{sunprog}
%list and explain current theories of sunquake generation
%making sure to highlight the different observables that can identify each mechanism, eg wlf = evidence of radiative backwarming  


The progenitors of sunquakes are still unknown and as a result this is an exciting area of research with discoveries still to be made. The general consensus, in terms of valid mechanisms that could cause this phenomenon is an area of contention, however the following progenitors are thought to be at least partly responsible. \\

\begin{itemize}
\item \textbf{Radiative backwarming} as a mechanism for producing sunquakes, was first put forward by \cite{2005ApJ...630.1168D} to account for a spatial correlation between seismic sources and white light emission from the lower atmosphere. The idea is that during a solar flare, high energy electrons and photons impulsively heat the photosphere producing white light emission \citep{1989SoPh..124..303M}. This causes an increase in radiation pressure in the photosphere which causes acoustic waves to propagate into the sub-photosphere. \\
     
\item \textbf{Sudden magnetic field reconfiguration} was first detailed by \cite{2008ASPC..383..221H}. Solar flares are violent physical processes that involve the evolution of magnetic fields and charged solar plasma. Due to the Lorentz force, it is possible for a magnetic field to impart a force on a charged material and vice versa. If, during a solar flare, the magnetic field close to the photosphere relaxes to a more horizontal alignment it can impart a force on the photospheric material resulting in the production of acoustic waves, which propagate into sub-photosphere. Thekey parameter for this mechanism seems to be that the field has to reconfigure in an sufficiently impulsive manner to generate enough force to induce seismic waves. \\

\item \textbf{Shocks} are a mechanism originally proposed in initial work by \cite{1995ESASP.376b.341K} and \cite{1998Natur.393..317K}, whereby a shock wave propagates from the upper-chromosphere down to the photosphere generating sub-photospheric acoustic waves. During a solar flare, particles and heat are directed down toward the chromosphere, at which point chromospheric material reacts with and increase in temperature. This increased temperature causes explosive ablation of chromospheric material both upward and downward. The downward component develops into a shock front carrying energy to the lower atmosphere, which can go on to heat the photosphere causing radiative backwarming \citep{1989SoPh..124..303M}. \\

\item \textbf{Direct proton collision}, is linked to observations by \cite{2007ApJ...664..573Z} where the sunquake was spatially aligned with $\gamma$-ray emission. $\gamma$-rays during a solar flare are an indicator of energetic protons being accelerated along a newly reconfigured magnetic field. Proton beams carry more momentum than electron beams and are able to penetrate into the lower atmosphere, depositing energy in the form of heat. If an energetic beam of protons makes it down to the photosphere, it can cause radiative backwarming \citep{1989SoPh..124..303M}, which in turn causes the sunquake as described above. \\

\end{itemize}




