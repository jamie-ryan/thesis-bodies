\begin{abstract}

Title: Lower Atmospheric Signatures Associated with a Seismically Active Solar Flare. 

Sunquakes represent the propagation of acoustic waves in the sub-photosphere, responding to an excitation of the photosphere during the impulsive phase of solar flares. The progenitors of sunquakes are thought to be either shocks, radiative backwarming, direct particle collision or sudden magnetic field reconfiguration. Each of these mechanisms relies on the transport of energy from the corona to the photosphere, and the physical conditions existing in the chromosphere such as magnetic configuration and density. To understand sunquakes and their relationship to solar flares, we need to understand how energy moves down through the solar atmosphere and the physical conditions that are present. Using data collected by RHESSI, IRIS and SDO HMI during the seismically active X1 solar flare observed in AR NOAA 12017 on the 29th of March 2014 at 17:46 UT, energy deposition in the chromosphere down through to the photosphere is quantified. The aim being, to discover which, if any, of the proposed generation methods for sunquakes are likely to be present in the event.   
\end{abstract}



