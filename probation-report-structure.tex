\documentclass[10pt]{article}
\usepackage{hyperref}
\usepackage[hmargin=1.25cm, vmargin=0.55cm]{geometry}
\usepackage{amsmath}
\usepackage{mathtools}
\usepackage{fancyhdr}
\usepackage{float}%Places the float at precisely the location in the LaTeX code...i.e, [H]
\usepackage{wrapfig}
\usepackage{graphicx}%includegraphics
%converts eps to pdf
%\usepackage{subcaption} %need to find subcaption.sty for this to work!
\usepackage{epstopdf}
\DeclareGraphicsExtensions{.pdf,.png,.jpg,.eps}
\renewcommand{\vec}[1]{\mathbf{#1}}
\usepackage[round]{natbib}  %use the "Natbib" style for the references in the Bibliography
%\usepackage{aastex}%defines journal abbreviations in bib file



\title{Probation Report Structure}
\author{Jamie Ryan \\ 
Mullard Space Science Laboratory \\
University College London \\
Surrey, RH13 6NL, UK\\
\href{mailto:jamie.ryan.14@ucl.ac.uk}{jamie.ryan.14@ucl.ac.uk}
\date{}}
\begin{document}
\maketitle


%SECTION::::::::::::::::::::ABSTRACT:::::::::::::::::::::::::::::::
%%%%%%%%%%%%%%%%%%%%%%%%%%%%%%%%%%%%%%%%%%%%%%%%%%%%%%%%%%%%%%%%%%%
%%%%%%%%%%%%%%%%%%%%%%%%%%%%%%%%%%%%%%%%%%%%%%%%%%%%%%%%%%%%%%%%%%%
\begin{abstract}
should the abstract go here???
\end{abstract}

%SECTION:::::::::::::::::::::::SUNQUAKES:::::::::::::::::::::::::::
%%%%%%%%%%%%%%%%%%%%%%%%%%%%%%%%%%%%%%%%%%%%%%%%%%%%%%%%%%%%%%%%%%%
%%%%%%%%%%%%%%%%%%%%%%%%%%%%%%%%%%%%%%%%%%%%%%%%%%%%%%%%%%%%%%%%%%%
\section{Sunquakes}
\subsection{An Introduction to Sunquakes}
Chronological soft intro; use some of the lit review,
making sure to emphasize the connection to solar flares;Look at early papers predicting sunquakes(Wolff 70s and Kosovichev and Zharkova);the importance of sunquakes
\subsection{Sunquake Model}
Intro to the physics of sunquakes + cartoon
\subsubsection{Sunquake Progenitors}
list and explain current theories of sunquake generation
making sure to highlight the different observables that can identify each mechanism, eg wlf = evidence of radiative backwarming  
\subsection{Observable Seismic Signatures}
Basics of helioseismology and challenges of observing acoustic emission
\subsubsection{Local Helioseismology}
use content from old report...maybe expand a little

%not sure if this section should be before Sunquake Progenitors?
%SECTION::::::::::::ERUPTIVE SOLAR FLARES::::::::::::::::::::::::::
%%%%%%%%%%%%%%%%%%%%%%%%%%%%%%%%%%%%%%%%%%%%%%%%%%%%%%%%%%%%%%%%%%%
%%%%%%%%%%%%%%%%%%%%%%%%%%%%%%%%%%%%%%%%%%%%%%%%%%%%%%%%%%%%%%%%%%%
\section{Eruptive Solar Flares}
\subsection{An Introduction to the Standard Eruptive Flare Model} 
standard solar flare model
cartoon
\subsection{Observing Solar Flares}
a bit about the the spacecraft and instrumentation providing the data for my research. Make sure to explain abbreviations
\subsubsection{Solar Atmosphere}
a bit about observing different layers of the atmosphere in different wavelengths...how do we know the altitudes of the emission?
can use modified solar atmosphere section (from old report)....rewrite pressure scale height...i.e,energy moving through the atmosphere has to traverse 9 pressure scale heights...what that means
\subsection{Magnetohydrodynamics of Solar Flares}
mhd maths and explanation..and what the induction equation physically means...maybe use a figure!!!
get to grips with the derivation, why is each assumption made?
can use mhd section from old report

%SECTION:LOWER ATMOSPHERIC SIGNATURES OF A SOLAR FLARE ASSOCIATED WITH SEISMICITY:::::::::::::::::::::
%%%%%%%%%%%%%%%%%%%%%%%%%%%%%%%%%%%%%%%%%%%%%%%%%%%%%%%%%%%%%%%%%%%%%%%%%%%%%%%%%%%%%%%%%%%%%%%%%%%%%%
%%%%%%%%%%%%%%%%%%%%%%%%%%%%%%%%%%%%%%%%%%%%%%%%%%%%%%%%%%%%%%%%%%%%%%%%%%%%%%%%%%%%%%%%%%%%%%%%%%%%%%
\section{Lower Atmospheric Signatures of a Solar Flare Associated with Seismicity}
use old report content + new content in presentation (inc full page plots but better/bigger axis labels)
\begin{itemize}
\item Abstract: Mark didn't like the abstract here
\item Background: Expand and de-itemize background (background should not be covering ground already in previous sections)
\item Observations: More detailed
\item Analysis: David Fanning Ladder plots. Analysis section should include more about techniques used to process the data.
\item Discussion
\item Future Work
\end{itemize}

\end{document}
