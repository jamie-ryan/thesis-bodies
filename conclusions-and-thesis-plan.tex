\section{Conclusions and Thesis Plan}
Calculations estimate that the sunquake is generated by an input momentum in the range $4.73{\times}10^{22} \leq p_{sqk} \geq 4.82{\times}10^{25}$ g cm s$^{-1}$. 50 to 100 keV HXR measurements show the impulsive phase of the flare occurs from 17:46 to 17:47. HXR contours and nonthermal power curve show that there is a energetic particle beam directly over the sunquake. When compared to the sunquake power the nonthermal electron beam has 1000 times more energy, which is enough to cause the sunquake. However, further inspection reveals that the electron beam has insufficient by five orders of magnitude in momentum to generate the sunquake. A proton beam is also considered, this is a theoretical upper limit calculation of the momentum carried by such a beam. It turns out that the proton beam is also lacking the momentum to cause the sunquake by four orders of magnitude. For these reasons it is unlikely that a nonthermal particle beam is capable of producing a sunquake in this event. Instead, nonthermal energy is deposited in the lower solar atmosphere giving rise to various emission. The Balmer continuum sampled from the sunquake location emits at a power that is thirteen orders of magnitude smaller than the power of the sunquake. Whilst radiative backwarming radiation momentum calculations yield a result that is seventeen orders of magnitude to small. These calculations performed on the Balmer continuum data show that there is not enough energy or momentum in the radiative backwarming to generate the sunquake. HMI continuum sampled from the sunquake location emits at a power which is lacking by twelve orders of magnitude when compared to that needed to produce the sunquake. Also HMI continuum emits approximately ten times more power than Balmer continuum, meaning it is unlikely that radiative backwarming is causing the white light flare. Energies integrated over the impulsive phase of the flare for Balmer and HMI continua show a similar results. Both being short of energy by ten orders of magnitude, reiterating that radiative backwarming is not generating the sunquake. There must be some other energy source involved in producing the acoustic source. 

From the analysis, it is clear that the sunquake is unlikely to be caused by either a particle baem or radiative backwarming, so what else could be causing the sunquake? \cite{2015ApJ...812...35M} show that there is substantial redshifted spectral lines during the impulsive phase that indicate the possible existence of shocks. Momentum associated with a downward shock is predicted to be .... Another possibility is sunquake production via a transient Lorentz force caused by the recinfiguring magnetic field. However, \cite{2014ApJ...796...85J} show that the change in magnetic field over the sunquake is not large enough to generate a sufficient Lorentz force capable of producing the sunquake. A purely theoretical idea could be in the form of MHD waves. Dissapation of Alfven waves \citep{1982SoPh...80...99E} has potential as a possible energy transport mechanism maybe capable of producing a sunquake \citep{2015ApJ...812...35M}. Energy in the form of Alfven waves could potentially propagate downward through the solar atmosphere, along the magnetic field. These waves are said to experience with little in the way of dissapation until reaching the temperature minimum region in the upper-photosphere. At which point Alfven waves are dissapated, heating photospheric plasma and causing white light flares. Observation of these waves is challenging, however there have been reports of associated signatures seen in the corona \citep{2009A&A...501L..15B}. 

%radiative backwarming conclusions
\paragraph{Thesis Plan}
Current Project) Lower Atmospheric Signatures in a Solar Flare Associated with Seismicity\\
Start: September 2014\\
Finish: April/May 2016\\

Project 2) Sunquakes; A Statistical Study\\
Start: April/May 2016\\
Finish: February 2017\\

The idea for this project is a statistical overview of many sunquake events, by developing new tools and using those developed in the first project energy. Using SDO AIA and HMI, physical properties such as energy, momentum, plasma velocity and magnetic lorentz force will be determined for each event, providing an extensive investigation into the nature of sunquakes. Using a combination of bith AIA and HMI allows the solar atmosphere to be sampled at various temperatures, which will provide information regarding energy transport from the corona to the phtosphere. This project will help to increase our knowledge of energy transport during solar flares and generation mechanisms for sunquakes. The project will also provide the statistics associated with a collection of many events and their properties. 

Project 3) Is Energy Dissipation of Alfven Waves in the Lower Solar Atmosphere Capable of Producing a Sunquake? \\
Start: April/May 2016\\
Finish: February 2017\\

The main aim is to search for observational signatures of Alfven waves during solar flares that generate sunquakes. The literature describes dissapation of Alfven waves as producing white light flares \citep{1982SoPh...80...99E, 2013AGUFMSH51A2091F}. Using SDO HMI the energy output of white light flares can be measured. Comparing HMI continuum energy measurements to energy carried by Alfven waves will provide much needed insight into these illusive MHD waves. Results from such a study stands to advance our understanding of the generation of both white light flares and sunquakes.   


