\section{Conclusions and Thesis Plan}

\paragraph{Conclusions}
%particle beam conclusions
50 to 100 keV HXR measurements show the impulsive phase of the flare occurs from 17:46 to 17:47. HXR contours and power curve  show that there is a nonthermal electron beam directly over the sunquake. 

The electron beam has/hasn't enough energy to cause the sunquake.

Even with the most idealistic of circumstances, the electron beam momentum $p_e$ is $\sim 10^{5}$ times smaller than the mometum required to produce the sunquake $p_{sqk}$. It is a similar conclusion for the theoretical proton beam momentum $p_{p}$, which is $10^{4}$ times smaller than $p_{sqk}$.

For these reasons it is unlikely that a nonthermal particle beam is capable of producing a sunquake in this event. Instead, nonthermal energy is deposited in the lower solar atmosphere giving rise to various emission. 

%radiative backwarming conclusions
Although there seems to be sufficient radiative backwarming occuring to cause the white light observed by HMI, it is evident from time integrated flux calculations of the Balmer continuum emission that there is insufficient radiative energy to generate the sunquake. 



\paragraph{Thesis Plan}
1) Lower Atmospheric Signatures in a Solar Flare Associated with Seismicity\\
Start: September 2014\\
Finish: April/May 2016\\

What's left to do:


\\
2) Sunquakes; A Statistical Study\\
Start: April/May 2016\\
Finish: February 2017\\

\\
3) Is Energy Dissipation of Alfven Waves in the Lower Solar Atmosphere Capable of Producing a Sunquake? \\
Start: April/May 2016\\
Finish: February 2017\\

\citep{1982SoPh...80...99E} first paper
\citep{2013AGUFMSH51A2091F} modern paper

