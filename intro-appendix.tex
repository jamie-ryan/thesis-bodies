\appendix
\appendixpage
\addappheadtotoc
%
% \begin{figure}%[H]
%   \begin{center}
%   \includegraphics[width=0.8\textwidth]{}
%   \end{center}
%   \caption{Left panel shows data over the quake location, right panel shows data over the ribbon location. From top to bottom, plots show lightcurves from IRIS Si IV, Mg II, Balmer wavelengths and Mg II wing, with the bottom panel showing the lightcurve from SDO HMI.}\label{lcseries-bold}
% \end{figure}

%insert figure showing ribbon coords oplot
%\graphicspath{ {~/PhD/Thesis/upgrade-plots/}}


%contains energy time plots for quake and ribbon locations
%also contains energy tables


%\includegraphics[trim=left bottom right top, clip]{file}
\subsection{Solar Magnetohydrodynamics}\label{MHD}
In the context of the Sun, and the importance of magnetic fields for processes such as solar flares, MHD builds on the following physical assumptions; a magnetic field can manipulate a plasma by exerting a force on it. Leading to the formation of structure or movement via acceleration; a magnetic field can store the energy required for later release as a solar flare; material wrapped in a magnetic field is thermally protected from it's surroundings; a magnetic field can act as a funnel for plasma and fast particles; and finally, a magnetic field can drive instabilities and support waves \citep{2003dysu.book.....D}.
 
\subsection{Induction Equation and the Magnetic Reynolds Number}
The induction equation expresses the time derivative of the magnetic field in terms of its diffusion $\eta\nabla^{2}\vec{B}$ and advection $\nabla\times(\vec{v}\times\vec{B})$ in the form, 


\begin{equation}\label{induction}
\frac{\partial \vec{B}}{\partial t}=\nabla\times(\vec{v}\times\vec{B})+\eta\nabla^{2}\vec{B}  
\end{equation}

where magnetic diffusivity $=\eta =\frac{1}{\mu_{0}\sigma}$. \\

The magnetic Reynolds number, $R_m$, is the ratio of diffusion to advection in a magnetic field. Where $l$, $v$ and $\sigma$ are typical length-scale, velocity and conductivity respectively, $R_m$ can be written,  

\begin{equation}\label{reynolds}
R_{m} = \frac{l_{0}v_{0}}{\eta} \sim \frac{\nabla\times(\vec{v}\times\vec{B})}{\eta\nabla^{2}\vec{B}}=\mu_{0}\sigma v l
\end{equation}

This ratio can be used to diagnose which part of the induction equation is dominating the MHD of the system. \\

If $R_m >> 1$ then advection is the major force changing the magnetic field, hence:
\begin{equation}\label{r>>1}
\frac{\partial \vec{B}}{\partial t}=\nabla\times(\vec{v}\times\vec{B})
\end{equation}
\\

Whereas, if $R_m << 1$ then diffusion is the dominant force acting on the magnetic field, leading to:
\begin{equation}\label{r<<1}
\frac{\partial \vec{B}}{\partial t}=\eta\nabla^{2}\vec{B}
\end{equation}

If diffusion is dominant, the magnetic field $\vec{B}$ will be varying on a length-scale $L_0$, and so will diffuse with velocity \ref{diffvel}, over the time-scale \ref{difftime} \citep{2003dysu.book.....D}.

\begin{equation}\label{diffvel}
v_d=\frac{L_0}{\tau_d} = \frac{\eta}{L_0}
\end{equation}


\begin{equation}\label{difftime}
\tau_d = \frac{L_{0}^{2}}{\eta}
\end{equation}

%check Foukal pg 125 in the pdf


%puts .tex file here
%\include{example}
\subsection{Plasma Beta}
The plasma $\beta$ is the ratio of plasma pressure, $p_{plasma}$, and magnetic pressure, $p_{mag}$. A measure of dominance of plasma or magnetic pressure in an MHD system can be expressed as, 

\begin{equation}\label{beta}
\beta=\frac{p_{plasma}}{p_mag} = \frac{2p_{0}\mu}{B_{0}^2}
\end{equation}

so if $\beta << 1$ then magnetic pressure is dominant and if $\beta >> 1$ then plasma pressure is dominant. In the context of the solar atmosphere, the magnetic domain of the corona has $\beta << 1$ and the dense plasma domain of the photosphere has $\beta >> 1$.

\subsection{Pressure Scale Height}
Another useful relationship to use is pressure scale height, where $g$ is acceleration due to gravity and $T_0$ is uniform temperature, pressure scale height, $H=\frac{P_0}{\rho_{0}g} = \frac{\Re T_{0}}{g}$. When combined with pressure, $p$ for a magneto static plasma with uniform temperature, $p=p_{0}\exp^{\frac{-z}{H}}$, where $z$ is altitude, we have a measure of the height, $H$, over which the pressure of a plasma fall off by a factor of $\exp$. 

\subsection{Flux Tubes}
Magnetic fields are made up of discreet bundles of magnetic flux known as \emph{flux tubes}. A magnetic flux tube can be thought of as cylindrical in geometry and containing magnetic field lines parallel in orientation to the length of the cylinder. The cross-sectional radius of the tube and magnetic field strength are both variant, magnetic flux contained within the tube however, is constant. Where $\vec{B}$ is magnetic field vector and $\vec{dS}$ is a cross-sectional surface element of the tube, flux $F$ follows the relationship.

\begin{equation}\label{fluxtube}       
F = \int_{S} \vec{B}.\vec{dS} \text{ [T cm}^{-2}]
\end{equation}



\subsection{Sunquake Power-Momentum Relationship}\label{sqk-ptoP-relation}
Equation \ref{sqk-momentum} is derived using the following steps. Where $F_{sqk}$ is the acoustic energy emitted by the sunquake area ($A_{sqk}$) per second, sunquake acoustic power is, 
\begin{equation}
P_{sqk} = F_{sqk} \; A_{sqk} = F_{sqk} \; l^2 \text{ [erg s}^{-1}]
\end{equation}\label{Psqk}
Rearranging equations \ref{sqk-momentum} and \ref{Psqk} for $l^{2}$ yields,
\begin{equation}
l^2 = \frac{P_{sqk}}{F_{sqk}} = \frac{p_{sqk}}{\rho \; v \; l} \text{ [cm}^{2}]
\end{equation}\label{l^2}
which can then be manipulated into the form shown in equation \ref{sqk-momentum}. 



