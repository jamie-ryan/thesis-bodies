\documentclass[10pt]{article}
\usepackage{hyperref}
\usepackage[hmargin=1.25cm, vmargin=0.55cm]{geometry}
\begin{document}
\title{Brief Statement}
\author{Jamie Ryan \\ 
Mullard Space Science Laboratory \\
University College London \\
Surrey, RH13 6NL, UK\\
\href{mailto:jamie.ryan.14@ucl.ac.uk}{jamie.ryan.14@ucl.ac.uk} \\
\date{\today}
}
\maketitle



To understand the mechanisms that control a large range of phenomena, from sunquakes to coronal mass ejections, we need to understand how energy moves through the entire solar atmosphere. In fact, the majority of the energy released by a flare is deposited in the lower solar atmosphere and manifests itself in the form of enhanced hard X-ray, UV and optical radiation. The radiative processes associated with impulsive flares span almost the entire range of wavelengths, including ultraviolet, white light continuum, x-rays, gamma rays and microwaves. The hard x-ray footpoints and UV ribbons observed in the chromosphere directly map to the reconfiguring magnetic fields during the flare, providing an important observational tool for understanding the magnetic reconnection process. How magnetic energy is transported from high altitudes in the solar corona, through many pressure scale heights to the lower atmosphere, is still poorly understood, and this is the primary focus of my research. \\

My PhD is primarily based on investigating how energy is released and transported in solar flares, both eruptive and non-eruptive, and in particular understanding the lower atmospheric signatures of the process, and how they relate to possible mechanisms for the production of local acoustic waves in the photosphere and sub-photosphere. IRIS provides invaluable data for this pursuit and being in the first year of my PhD, I believe that this is the perfect time to enrol on this workshop. The chromosphere is a very dynamic region due to the changing influences (on the residing plasma) of gas and magnetic pressure. This part of the solar atmosphere effectively controls how energy is released into the upper layers, such as the transition region and corona, and how it is filtered back down to the photosphere. In order to grasp the physics governing the observed motions of plasma within the chromosphere, it is vital to understand the physical parameters dictating such events; IRIS has the unique ability to provide both imagery and spectroscopic data needed to achieve this goal. \\

IRIS is capable of capturing information from the Sun in exquisite spatial and temporal resolution, allowing the morphologies and physical properties of features to be more accurately observed. The ability to couple intensity maps with spectroscopic observations means that the tracking of a feature through space and time can be linked directly to particular line shapes, spectral widths and displacements, allowing diagnostics of the physical parameters associated with energy deposition processes, such as waves and shocks, to become accessible. For this reason, the chromospheric MgII h and k lines observed by IRIS should prove to be a vital tool for my research. However, interpreting these spectroscopic data is a non-trivial matter. Attendance at the IRIS-4 Workshop is crucial for equipping me with the understanding of how to work with such data, and will prove critical for the success of my research.\\
\\

Thank you for your consideration



  


\end{document}
