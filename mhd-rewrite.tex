%ADD FROZEN-IN THEOREM
%FLUX ROPE FORMATION?



%%%%%%%%%%%%%%%%%%%%%%%%%%MHD of solar flares%%%%%%%%%%%%%%%%%%%%%%%%%%%5
\subsection{Solar Magnetohydrodynamics}\label{MHD}
Most structures observed in the solar atmosphere are a direct result of interplay between plasma and the Sun's dynamic magnetic field. Understanding the relationship between a magnetic field and a plasma is important for describing many observed phenomena. Magnetohydrodynamics (MHD) is a method to determine the continuous macroscopic behaviour of plasma in a magnetic field, thus individual particles are not considered \citep{1982soma.book.....P}. A plasma can be treated as a continuous material if distances between particles are much larger than the mean-free path, or larger than the ion gyro-radius. Where $T$ is plasma temperature, $\lambda_{MFP}$ is mean-free path and $n$ is number of particles in the plasma, equation \ref{meanfreepath} describes the mean-free path. 

\begin{equation}\label{meanfreepath}
\lambda_{MFP}\approx300(\frac{T}{10^{6}K})^{2}(\frac{n}{10^{17}m^{-3}})^{-1}m
\end{equation}

Equation \ref{iongyroradius} is the relationship between particle mass $m$, velocity perpendicular to the magnetic field $v_{\bot}$, charge $q$ and magnetic field strength $B$  which governs the circular motion of a charged particle around a uniform magnetic field or the ion gyro radius. 

\begin{equation}\label{iongyroradius}
r_{g}=\frac{mv_{\bot}}{|q|B}
\end{equation}

In the context of the Sun, and the importance of magnetic fields for processes such as solar flares, MHD builds on the following physical assumptions; a magnetic field can manipulate a plasma by exerting a force on it. Leading to the formation of structure or movement via acceleration; a magnetic field can store the energy required for later release as a solar flare; material wrapped in a magnetic field is thermally protected from it's surroundings; a magnetic field can act as a funnel for plasma and fast particles; and finally, a magnetic field can drive instabilities and support waves \citep{2003dysu.book.....D}.

%\subsubsection{Magnetic Structure}\label{MHDeqns}
\subsubsection{Flux Tubes}
Magnetic fields are made up of discreet bundles of magnetic flux known as \emph{flux tubes}. A magnetic flux tube can be thought of as cylindrical in geometry and containing magnetic field lines parallel in orientation to the length of the cylinder. The cross-sectional radius of the tube and magnetic field strength are both variant, magnetic flux contained within the tube however, is constant. Where $\vec{B}$ is magnetic field vector and $\vec{dS}$ is a cross-sectional surface element of the tube, flux $F$ follows the relationship.

\begin{equation}\label{fluxtube}       
F = \int_{S} \vec{B}.\vec{dS}
\end{equation}


\subsubsection{MHD Equations}
The basics of MHD are a built from a combination of Maxwell's electromagnetic equations, material equations, Ohm's law and the fluid dynamics relations. Please see section \ref{mhdeqns} in the appendices. Maxwell's equations and Ohm's law describe electromagnetism in terms of magnetic induction $\vec{B}$, magnetic permeability of free space $\mu$, electric field $\vec{E}$, electrical permittivity of free space $\epsilon_{0}$, charge density $\rho_{c}$ and electric current density $\vec{j}$ \\ 

 
\subsubsection{Induction Equation}
The induction equation expresses the time derivative of the magnetic field in terms of its diffusion $\eta\nabla^{2}\vec{B}$ and advection $\nabla\times(\vec{v}\times\vec{B})$ in the form, 

\textbf{Induction Equation}
\begin{equation}\label{induction}
\frac{\partial \vec{B}}{\partial t}=\nabla\times(\vec{v}\times\vec{B})+\eta\nabla^{2}\vec{B}  
\end{equation}

where magnetic diffusivity $=\eta =\frac{1}{\mu_{0}\sigma}$. \\


\textbf{The Magnetic Reynolds Number}
The magnetic Reynolds number, $R_m$, is the ratio of diffusion to advection in a magnetic field. Where $l$, $v$ and $\sigma$ are typical length-scale, velocity and conductivity respectively, $R_m$ can be written,  

\begin{equation}\label{reynolds}
R_{m} = \frac{l_{0}v_{0}}{\eta} \sim \frac{\nabla\times(\vec{v}\times\vec{B})}{\eta\nabla^{2}\vec{B}}=\mu_{0}\sigma v l
\end{equation}

This ratio can be used to diagnose which part of the induction equation is dominating the MHD of the system. \\

If $R_m >> 1$ then advection is the major force changing the magnetic field, hence:
\begin{equation}\label{r>>1}
\frac{\partial \vec{B}}{\partial t}=\nabla\times(\vec{v}\times\vec{B}),
\end{equation}
\\

Whereas, if $R_m << 1$ then diffusion is the dominant force acting on the magnetic field, leading to:
\begin{equation}\label{r<<1}
\frac{\partial \vec{B}}{\partial t}=\eta\nabla^{2}\vec{B},
\end{equation}

If diffusion is dominant, the magnetic field $\vec{B}$ will be varying on a length-scale $L_0$, and so will diffuse with velocity \ref{diffvel}, over the time-scale \ref{difftime}.

\begin{equation}\label{diffvel}
v_d=\frac{L_0}{\tau_d} = \frac{\eta}{L_0}
\end{equation}


\begin{equation}\label{difftime}
\tau_d = \frac{L_{0}^{2}}{\eta}
\end{equation}

%check Foukal pg 125 in the pdf
\citep{2003dysu.book.....D}

%puts .tex file here
%\include{example}
\subsubsection{Plasma Beta}
The plasma $\beta$ is the ratio of plasma pressure, $p_{plasma}$, and magnetic pressure, $p_{mag}$. A measure of dominance of plasma or magnetic pressure in an MHD system can be expressed as, 

\begin{equation}\label{beta}
\beta=\frac{p_{plasma}}{p_mag} = \frac{2p_{0}\mu}{B_{0}^2}
\end{equation}

so if $\beta << 1$ then magnetic pressure is dominant and if $\beta >> 1$ then plasma pressure is dominant. In the context of the solar atmosphere, the magnetic domain of the corona has $\beta << 1$ and the dense plasma domain of the photosphere has $\beta >> 1$.

\subsubsection{Pressure Scale Height}
Another useful relationship to use is pressure scale height, where $g$ is acceleration due to gravity and $T_0$ is uniform temperature, pressure scale height, $H=\frac{P_0}{\rho_{0}g} = \frac{\Re T_{0}}{g}$. When combined with pressure, $p$ for a magneto static plasma with uniform temperature, $p=p_{0}\exp^{\frac{-z}{H}}$, where $z$ is altitude, we have a measure of the height, $H$, over which the pressure of a plasma fall off by a factor of $\exp$. 

