%ADD FROZEN-IN THEOREM
%FLUX ROPE FORMATION?



%%%%%%%%%%%%%%%%%%%%%%%%%%MHD of solar flares%%%%%%%%%%%%%%%%%%%%%%%%%%%5
\subsection{Solar Magnetohydrodynamics}\label{MHD}
Most structures observed in the solar atmosphere are a direct result of interplay between plasma and the Sun's dynamic magnetic field. Understanding the relationship between a magnetic field and a plasma is important for describing many observed phenomena. Magnetohydrodynamics (MHD) is a method to determine the continuous macroscopic behaviour of plasma in a magnetic field, thus individual particles are not considered \citep{1982soma.book.....P}. A plasma can be treated as a continuous material if distances between particles are much larger than the mean-free path, or larger than the ion gyro-radius. Where $T$ is plasma temperature, $\lambda_{MFP}$ is mean-free path and $n$ is number of particles in the plasma, equation \ref{meanfreepath} describes the mean-free path. 

\begin{equation}\label{meanfreepath}
\lambda_{MFP}\approx300(\frac{T}{10^{6}K})^{2}(\frac{n}{10^{17}m^{-3}})^{-1}m
\end{equation}

Equation \ref{iongyroradius} is the relationship between particle mass $m$, velocity perpendicular to the magnetic field $v_{\bot}$, charge $q$ and magnetic field strength $B$  which governs the circular motion of a charged particle around a uniform magnetic field or the ion gyro radius. 

\begin{equation}\label{iongyroradius}
r_{g}=\frac{mv_{\bot}}{|q|B}
\end{equation}

In the context of the Sun, and the importance of magnetic fields for processes such as solar flares, MHD builds on the following physical assumptions; a magnetic field can manipulate a plasma by exerting a force on it. Leading to the formation of structure or movement via acceleration; a magnetic field can store the energy required for later release as a solar flare; material wrapped in a magnetic field is thermally protected from it's surroundings; a magnetic field can act as a funnel for plasma and fast particles; and finally, a magnetic field can drive instabilities and support waves \citep{2003dysu.book.....D}.

\subsubsection{MHD Equations}\label{MHDeqns}
Magnetic fields are made up of discreet bundles of magnetic flux known as \emph{flux tubes}. A magnetic flux tube can be thought of as cylindrical in geometry and containing magnetic field lines parallel in orientation to the length of the cylinder. The cross-sectional radius of the tube and magnetic field strength are both variant, magnetic flux contained within the tube however, is constant. Where $\vec{B}$ is magnetic field vector and $\vec{dS}$ is a cross-sectional surface element of the tube, flux $F$ follows the relationship.

\begin{equation}\label{fluxtube}       
F = \int_{S} \vec{B}.\vec{dS}
\end{equation}


The basics of MHD are a built from a combination of Maxwell's electromagnetic equations, material equations, Ohm's law and the fluid dynamics relations. Maxwell's equations and Ohm's law describe electromagnetism in terms of magnetic field $\vec{H}$, magnetic induction $\vec{B}$, magnetic permeability of free space $\mu$, electric field $\vec{E}$, electric displacement $\vec{D}$, electrical permittivity of free space $\epsilon$, charge density $\rho_{c}$ and electric current density $\vec{j}$ 


\begin{equation}\label{max1:ampere}
\nabla\times\vec{H}=\vec{j}+\frac{\partial \vec{D}}{\partial t}
\end{equation}


\begin{equation}\label{max2:faraday}
\nabla\times\vec{E}=-\frac{\partial \vec{B}}{\partial t}
\end{equation}


\begin{equation}\label{max3:gauss}
\nabla\cdot\vec{D}=\rho_{c}
\end{equation}


\begin{equation}\label{max4:nomonopole}
\nabla\cdot\vec{B}=0
\end{equation}

with Ohm's law as,

\begin{equation}\label{ohmslaw}
\vec{E}=\frac{\vec{j}}{\sigma}
\end{equation}

where $\sigma$ is electrical conductivity. The fluid dynamics relations are written in terms of density $\rho$, velocity $\vec{v}$, time $t$, pressure $p$, gas constant $\Re$ and temperature $T$. 
%$\frac{d}{dt}=\frac{\partial}{\partial t}+ \vec{v}\cdot\nabla$, sometimes reffered to as the material derivative, is a way of stating the rate of change with respect to time of the  

\begin{equation}\label{motion}
\rho\frac{d\vec{v}}{dt}=-\nabla p
\end{equation}
 
\begin{equation}\label{masscontinuity}
\frac{d\rho}{dt}+\rho\nabla\cdot\vec{v}=0
\end{equation}

\begin{equation}\label{perfectgaslaw}
p=\Re\rho T
\end{equation}

Equation \ref{motion} is the eqn. of motion describing how the forces are exerted on the fluid are equal to the negative gradient of plasma pressure. Equation \ref{masscontinuity} is the eqn. of mass continuity (mass is conserved), whilst equation \ref{perfectgaslaw} is the perfect gas law relating plasma pressure, density and temperature. The fluid dynamics equations are re-written in terms of MHD by building on the following assumptions: Plasma in a magnetic field experiences the Lorentz force $(\vec{j}\times\vec{B})$, such that a plasma volume element $dV$ carrying a current density $\vec{j}$ per unit volume has the force $\vec{j}dV\times\vec{B}$ exerted on it in a perpendicular direction to the magnetic field. In order to cater for the extra force, equation \ref{motion} has the term $\vec{j}\times\vec{B}$ added to the right hand side.   
Ohm's law states that an electric field in the plasma's frame of reference is proportional to the current, however, the total electric field associated with the movement of plasma is $\vec{E}+\vec{v}\times\vec{B}$ ($\vec{E}$ is the electric field acting on the plasma at rest) therefore, equation \ref{ohmslaw} has the term $\vec{v}\times\vec{B}$ added to it's left hand side.The displacement current term $\frac{\partial \vec{D}}{\partial t}$ in equation \ref{max1:ampere} is neglected due to plasma speeds being much slower than the speed of light.

\subsubsection{Induction Equation}\label{inductioneqn}  
Rewriting equations \ref{max1:ampere}, \ref{max2:faraday} and \ref{ohmslaw} using the assumptions from section \ref{MHDeqns}, with $\nabla\cdot\vec{B} = 0$,

\begin{equation}\label{new1}
\vec{j}=\nabla\times\frac{\vec{B}}{\mu}
\end{equation}

\begin{equation}\label{new2}
\frac{\partial \vec{B}}{\partial t} = - \nabla\times\vec{E}
\end{equation}

\begin{equation}\label{new3}
\vec{E}=-\vec{v}\times\vec{B}+\frac{\vec{j}}{\sigma}
\end{equation}

then substituting \ref{new1} solved for $\vec{j}$ into \ref{new2} we have,

\begin{equation}\label{new4} 
\vec{E}=-\vec{v}\times\vec{B}+\nabla\times\frac{\vec{B}}{\eta} 
\end{equation}

where magnetic diffusivity $=\eta =\frac{1}{\mu\sigma}$.

Substituting \ref{new4}, solved for $\vec{E}$, into \ref{new2},

\begin{equation}\label{new5}
\frac{\partial \vec{B}}{\partial t}=-\nabla\times(-\vec{v}\times\vec{B} + \nabla\times\vec{B}\eta)
                                   =\nabla\times(\vec{v}\times\vec{B})-\nabla\eta\times(\nabla\times\vec{B})
\end{equation}

using the vector identity, $\nabla\times(\nabla\times\vec{A})=\nabla(\nabla\cdot\vec{A})-\nabla^{2}\vec{A}$ the last term in equation \ref{new5} becomes $\nabla(\nabla\cdot\vec{B})-\nabla^{2}\vec{B}$, of which the $\nabla\cdot\vec{B}$ term reduces to zero due to equation \ref{max4:nomonopole}, thus we are left with the induction equation \ref{induction} below. 



\begin{equation}\label{induction}
\frac{\partial \vec{B}}{\partial t}=\nabla\times(\vec{v}\times\vec{B})+\eta\nabla^{2}\vec{B}  
\end{equation}

This equation \ref{induction} can be used to determine $\vec{B}$ if $\vec{v}$ is known. The magnetic Reynolds number,

\begin{equation}\label{reynolds}
R_{m} = \frac{l_{0}v_{0}}{\eta} \sim \frac{\nabla\times(\vec{v}\times\vec{B})}{\eta\nabla^{2}\vec{B}}
\end{equation}

is an approximation of the ratio between the first and second terms on the RHS of the induction equation \ref{induction} if $v_0$ and $l_0$ are typical of the velocity and length-scales over which the system is changing. This ratio can be used to diagnose which part of the induction equation is dominating the MHD of the system. For example if $R_m >> 1$ then the $\nabla\times(\vec{v}\times\vec{B})$ term is large and the second term is negligible, so the induction equation becomes,

\begin{equation}\label{r>>1}
\frac{\partial \vec{B}}{\partial t}=\nabla\times(\vec{v}\times\vec{B}),
\end{equation}

thus induction is dominant and Ohm's law \ref{ohmslaw} becomes $\vec{E} +\vec{v}\times\vec{B}$ meaning the total electric field becomes zero.



If $R_m << 1$ then the $\eta\nabla^{2}\vec{B}$ term is large meaning the induction equation becomes,
\begin{equation}\label{r<<1}
\frac{\partial \vec{B}}{\partial t}=\eta\nabla^{2}\vec{B},
\end{equation}

thus diffusion is dominant so the magnetic field $\vec{B}$ will be varying on a length-scale $L_0$, and so will diffuse with velocity \ref{diffvel}, over the time-scale \ref{difftime}.

\begin{equation}\label{diffvel}
v_d=\frac{L_0}{\tau_d} = \frac{\eta}{L_0}
\end{equation}


\begin{equation}\label{difftime}
\tau_d = \frac{L_{0}^{2}}{\eta}
\end{equation}

%check Foukal pg 125 in the pdf
\citep{2003dysu.book.....D}

%puts .tex file here
%\include{example}
\subsubsection{Plasma Beta}
Another way to write the relationship between $\vec{B}$ and $\vec{v}$ is the equation of motion \ref{motion} modified by the Lorentz force.

\begin{equation}\label{motion1}
\rho\frac{d\vec{v}}{dt}=-\nabla p+\vec{j}\times\vec{B}
\end{equation}

The pressure gradient $-\nabla p$ describes the change from high to low plasma pressure. The Lorentz force, $\vec{j}\times\vec{B}$ acts perpendicularly to the magnetic field, therefore accelerated plasma in a direction parallel to the magnetic field is caused by other forces. If equation \ref{new1} is substituted into the Lorentz force,
\begin{equation}\label{lorentzsub1}
\vec{j}\times\vec{B} = (\nabla\times\frac{\vec{B}}{\mu})\times\vec{B}
\end{equation}

then using the triple vector identity, 
\begin{equation}
\frac{1}2{}\nabla(\vec{B}\cdot\vec{B})=\vec{B}\times(\nabla\times\vec{B})+(\vec{B}\cdot\nabla)\vec{B}
\end{equation}

equation \ref{lorentzsub1} becomes,
\begin{equation}
\vec{j}\times\vec{B} = (\vec{B}\cdot\nabla)\frac{\vec{B}}{\mu}-\nabla(\frac{B^2}{2\mu}) 
\end{equation}

therefore \ref{motion1} becomes:

\begin{equation}\label{maghydstat}
\rho\frac{d\vec{v}}{dt}= -\nabla p + (\vec{B}\cdot\nabla)\frac{\vec{B}}{\mu}-\nabla(\frac{B^2}{2\mu}) 
\end{equation}

Because $-\nabla(\frac{B^2}{2\mu}$ is in the same form as $-\nabla p$ it can be said to be the \emph{magnetic pressure}, providing a force pointing from high to low magnetic pressure as $B^2$ changes with position.A measure of dominance of plasma or magnetic pressure in the motion of plasma is known as the plasma beta, 

\begin{equation}\label{beta}
\beta=\frac{p_{plasma}}{p_mag} = \frac{2p_{0}\mu}{B_{0}^2}
\end{equation}

so if $\beta << 1$ then magnetic pressure is dominant and if $\beta >> 1$ then plasma pressure is dominant.

Another useful relationship to use is pressure scale height, where $g$ is acceleration due to gravity and $T_0$ is uniform temperature, pressure scale height, $H=\frac{P_0}{\rho_{0}g} = \frac{\Re T_{0}}{g}$. When combined with pressure, $p$ for a magneto static plasma with uniform temperature, $p=p_{0}\exp^{\frac{-z}{H}}$, where $z$ is altitude, we have a measure of the height, $H$, over which the pressure of a plasma fall off by a factor of $\exp$. 

