\documentclass[11pt]{article}
\usepackage[utf8]{inputenc}
\usepackage[english]{babel}
\usepackage{aas_macros}
\usepackage{hyperref}
\usepackage{appendix}
%\usepackage[hmargin=1.5cm, vmargin=0.55cm]{geometry}
\usepackage[hmargin=1.5cm, vmargin=1.5cm]{geometry}
\usepackage{amsmath}
\usepackage{caption}
\usepackage{mathtools}
\usepackage{fancyhdr}
\usepackage{float}%Places the float at precisely the location in the LaTeX code...i.e, [H]
\usepackage{wrapfig}
\usepackage{rotating}
\usepackage{pdfpages}
%\usepackage[showframe]{geometry}
%for quotes
%\epigraphfontsize{\small\itshape}
%\setlength\epigraphwidth{8cm}
%\setlength\epigraphrule{0pt}


\usepackage[round]{natbib}  %use the "Natbib" style for the references in the Bibliography
%\usepackage{aastex}%defines journal abbreviations in bib file
%\newcommand{name}[num]{definition}



%\title{Lower Atmospheric Signatures During Solar Flares \\ Associated with Seismicity}
%\author{Jamie Ryan \\
%Mullard Space Science Laboratory \\
%University College London \\
%Surrey, RH13 6NL, UK\\
%\href{mailto:jamie.ryan.14@ucl.ac.uk}{jamie.ryan.14@ucl.ac.uk}
%\date{}}

%\pagestyle{headings}
%\setcounter{page}{1}

\begin{document}

\begin{sidewaystable}[h]
\tiny
\centering
\begin{tabular}{|c|c|c|c|}
\hline
Time Interval & Process  & Energy [erg] & Momentum [g cm s$^{-1}$] \\
\hline
29-Mar-2014 17:46:18 to 17:46:34 & Full Disc Spec 10 - 90 keV HXR & 4.80e+30 & 2.48e+23 \\
 & North Ribbon Img Spec 10 - 90 keV HXR & 4.06e+28 & 2.10e+21 \\
 & South Ribbon Img Spec 10 - 90 keV HXR & 1.19e+29 & 6.16e+21 \\
 & North Ribbon Radiative Backwarming & 6.05E+25 & 2.02E+15 \\
 & South Ribbon Radiative Backwarming QK Location 1 & 8.81E+26 & 2.94E+16 \\
 & South Ribbon Radiative Backwarming QK Location 2 & 5.63E+26 & 1.88E+16 \\
\hline
29-Mar-2014 17:46:50 to 17:47:06 & Full Disc Spec 10 - 90 keV HXR & 4.09e+30 & 2.11e+23 \\
 & North Ribbon Img Spec 10 - 90 keV HXR & 3.76e+28 & 1.94e+21 \\
 & South Ribbon Img Spec 10 - 90 keV HXR & 1.07e+29 & 5.52e+21 \\
 & North Ribbon Radiative Backwarming & 6.05E+25 & 2.02E+15 \\
 & South Ribbon Radiative Backwarming QK Location 1 & 8.81E+26 & 2.94E+16 \\
 & South Ribbon Radiative Backwarming QK Location 2 & 5.63E+26 & 1.88E+16 \\
\hline
29-Mar-2014 17:47:06 to 17:47:22 & Full Disc Spec 10 - 90 keV HXR & 4.58e+30 & 2.36e+23 \\
 & North Ribbon Img Spec 10 - 90 keV HXR & 2.34e+28 & 1.21e+21 \\
 & South Ribbon Img Spec 10 - 90 keV HXR & 1.72e+29 & 8.86e+21 \\
 & North Ribbon Radiative Backwarming & 4.80E+25 & 1.60E+15 \\
 & South Ribbon Radiative Backwarming QK Location 1 & 2.78E+26 & 9.28E+15 \\
 & South Ribbon Radiative Backwarming QK Location 2 & 3.33E+26 & 1.11E+16 \\
\hline
29 Mar 2014 17:46 & Sunquake & 1.30E+26 & 6.02E+22 \\
\hline
\end{tabular}
\caption{Energy and momentum values calculated for full disc nonthermal hard x-rays associated with electron beams; nonthermal hard x-rays associated with electron beams for each footpoint; radiative backwarming over the sunquake areas and in the northern ribbon; the sunquake. QK locations 1 and 2 refer to coordinates 518.5, 264.0 and 519.0, 262.0 respectively. These coordinates are quoted by Judge et al 2014 and Matthews et al 2015.}\label{ribenergytab}
\end{sidewaystable}


\label{Bibliography}
\lhead{\emph{Bibliography}}  % Change the left side page header to "Bibliography"
%\bibliographystyle{unsrtnat}  % Use the "unsrtnat" BibTeX style for formatting the Bibliography
\bibliographystyle{plainnat}%abbrv}
\bibliography{../Bibliography/Bibliography}  % The references (bibliography) information are stored in the file named "Bibliography.bib"
%\bibliography{Bibliography}
\end{document}
