\begin{abstract}
Sunquakes represent the propagation of acoustic waves in the sub-photosphere, responding to an excitation of the photosphere during the impulsive phase of solar flares. The progenitors of sunquakes are thought to be either shocks, radiative backwarming, direct particle collision or sudden magnetic field reconfiguration. Each of these mechanisms relies on the transport of energy from the corona to the photosphere, and the physical conditions existing in the chromosphere such as magnetic configuration and density. To understand sunquakes and their relationship to solar flares, we need to understand how energy moves down through the solar atmosphere and the physical conditions that are present. An X1 solar flare with associated sunquake was observed in active region NOAA 12017 on the 29th of March 2014 at 17:46 UTC, by multiple spacecraft, including SDO (HMI), IRIS and RHESSI. Lightcurves of the flare emission from the photosphere, chromosphere and transition region are analysed providing information about the deposition of energy at different altitudes in the solar atmosphere. Hard X-ray footpoints of coronal loops are shown to align well with an area associated with maximum acoustic power. Balmer continuum emission aligned with maximum acoustic power is shown to increase during the flare, indicating the existence of hydrogen recombination continua in the chromosphere possibly leading to radiative backwarming of the photosphere. 
\end{abstract}
