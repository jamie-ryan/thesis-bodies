%Mark says: This should describe the aims of your research study and why the research problem is important. It should also set the scene for everything that follows.

%Lucie says: 
%*Fact check my %mythology, %science-history and %modern-day-view paragraphs.
%*Fusion
%*Structure
%*Energy transport from core to photosphere
%*Atmosphere (show classic T, rho plot)
%*Magnetic field (permeates entire Sun ands atmosphere)
%   *a couple of statements about internal dynamo
%   *field transported by ("frozen-in") bouyant plasma in the convection zone (flux tubes created by dynamo)
%   *Sunspots to coronal loops
%*Magnetic activity
%   *reconnection
%   *cme
%   *flares   
%*Sunquakes
%*My science question

\section{Introduction}

\epigraphfontsize{\small\itshape}
%insert quote
\epigraph{``So you run and you run to catch up with the sun but it's sinking \\
Racing around to come up behind you again. \\
The sun is the same in a relative way but you're older, \\
Shorter of breath and one day closer to death."}{--- \textup{Roger Waters} Time, Darkside of the Moon, Pink Floyd}


%mythology (pre-science)
Early scrawlings on cave walls and writings by later societies show that the Sun has been an object of mystery and mythology for much of human history. Many pre-scientific civilisations worshipping this all powerful bringer of life, often in the form of a female deity. For instance, Norse society beleived that the Sun or Sól, rode her horse drawn chariot through the sky, in a daily race against pusuing darkness which has taken the form of ravenous wolves. It was thought that if Sól was overtaken by these dark wolves then cosmic chaos would insue, with the end result being the destruction of all that humans and the gods held dear. This particular fable shows how the Sun has been interwoven in to our explanation of the world around us, with, "what if the Sun does not rise?" being one of our oldest philisophical questions. To some extent it is thought that, such beliefs still permeate todays society, in fact, theologan scholars speculate that many of the modern forms of religion are probably the result of a rewriting of solar mythology.  


%science history
Other than being held as an all powerful omnipitant being, the Sun has fascinated inquisitive minds for much of history and there are documented examples of solar study, going back thousands of years. From as early as 2000 B.C., the Chinese were able to predict when a solar eclipse would occur. Moving forward around 1500 years to 350 B.C., and some of the first sunspot observations are made by Aristotles student, Theophrastus. These early examples of the Sun occupying the minds of the days critical thinkers surely makes the study of the Sun one of the oldest scientific pursuits.      


%modern day view
In comparison, scientists today have an exquisitely detailed, spaceborn view of the Sun. This has lead to an understanding of many of the physical processes that dictate the dynamics and life cycle of our parent star. It is now accepted that the Sun is a 4.5 billion year old G-type, main sequence star in a state of hydrostatic equilibrium. Powered by nuclear fusion, the Sun emits around $3.86\times10^{26}$ $ Joules of energy per second, which is enough to meet the world's annual power consumption a million times over. This energy is generated in the solar core, by nuclear decays that lead to chain reactions, releasing photons that can take approximately 100,000 years to travel from the core to the photosphere. The Sun is able to fuse atoms in this way because of the huge pressures generated within the core caused by the strong gravitational field. The Sun and all main sequence stars are in a state of hydrostatic equilibrium, whereby outward gas pressure is balanced by the gravity generated by the stars mass. At some point, probably in around 4.5 billion years, the Sun will start to run out of Hydrogen fuel, at which point nuclear fusion in the core will cease. The gravitation field will be weakened due to this lack of mass in the core. Once the Sun's gravity is too weak to resist internal gas pressure the star will start to expand and eventually shed it's outer layers forming a planetary nebula.


So, other than nurturing the intelligent life that occupies this planet, the Sun is one unspectactular star in a 100 billion that make up our galaxy.


