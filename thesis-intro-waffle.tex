\section{Introduction}

\epigraphfontsize{\small\itshape}
%insert quote
\epigraph{``So you run and you run to catch up with the sun but it's sinking \\
Racing around to come up behind you again. \\
The sun is the same in a relative way but you're older, \\
Shorter of breath and one day closer to death."}{--- \textup{Roger Waters} Time, Darkside of the Moon, Pink Floyd}


%mythology (pre-science)
Early scrawlings on cave walls and writings by later societies show that the Sun has been an object of mystery and mythology for much of human history. Many pre-scientific civilisations worshipping this all powerful bringer of life, often in the form of a female deity. For instance, Norse society beleived that the Sun or Sól, rode her horse drawn chariot through the sky, in a daily race against pusuing darkness in the form of wolves. It was thought that if Sól was overtaken by these dark wolves then cosmic chaos would insue, with the end result being the destruction of all that humans and the gods held dear. This particular fable shows how the Sun has been interwoven in to our explanation of the world around us, with, "what if the Sun does not rise?" being one of our oldest philisophical questions. In fact, to some extent, such beliefs still permeate todays society, with the modern word "Sun" being derived from Sól's alterante aliases, Sunna and Frau Sunne. Many theologan scholars tend to assume that many of the modern forms of religion are the result of a rewriting of solar mythology, the difference being that the emphasis is switched from the Sun to deities assuming a more recognisable human form.  


%science history
Other than being held as an all powerful omnipitant being, the Sun has fascinated inquisitive minds for much of human history. There are documented examples, going back thousands of years of such study. From as early as 2000 B.C., the Chinese were able to predict when a solar eclipse would occur. Moving forward around 1500 years to 350 B.C., and some of the first sunspot observations are made by Aristotles student, Theophrastus. These early examples of the Sun occupying the minds of the days critical thinkers surely makes the study of the Sun one of the oldest scientific pursuits.      


%modern day view
In comparison, scientists today have an exquisitely detailed, spaceborn view of the Sun. This has lead to an understanding of many of the physical processes that dictate the dynamics and life cycle of our parent star. It is now accepted that the Sun is a 4.5 billion year old G-type, main sequence star. Other than nurturing the intelligent life that occupies this planet, the Sun is one unspectactular star in a 100 billion that make up our galaxy.

